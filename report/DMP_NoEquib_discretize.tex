\section{Discretization}
\subsection{Time, Part I}
In a single time step $0\leq t\leq t_1$, the opacity and specific heat are
treated explicitly at $t=0$ as $\sigma_0$ and $c_{v0}$ respectively.  Next, it
is useful
to define a few variables and constants:
\begin{subequations}\label{defs}
\begin{align}
B(\nu,T) & =\mbox{Planck spectrum}
  =\frac{15ac}{4\pi^5}\frac{\nu^3}{e^{\nu/T}-1},\\
b(\nu,T) & =\mbox{normalized Plank spectrum}
  =\frac{15}{\pi^4T^4} \frac{\nu^3}{e^{\nu/T}-1}, \\
\sigma(\nu,T) & =\mbox{absorption opacity}=\frac{\gamma}{\nu^3}(1-e^{-\nu/T}),\\
\sigma_p &
  =\int_0^\infty b(\nu,T_0)\sigma(\nu,T_0)d\nu=\frac{15\gamma}{\pi^4T_0^3},\\
f_0 & =\mbox{Fleck factor}=\frac{1}{1+\beta_0\sigma_0c\Delta_t},\\
\beta_0 & =\frac{\partial aT_0^4/\partial T}{\partial
  c_{v0}T_0/\partial T}=\frac{4aT_0^3}{c_{v0}},\\
a & =\mbox{radiation constant}=0.01372 \frac{\mbox{jk}}{\mbox{cm$^3$keV$^4$}},\\
c & =\mbox{speed of light}=299.792458 \mbox{cm $/$ sh}.
\end{align}
\end{subequations}
With Eqs. \eqref{defs} defined and moving to a 1-dimensional geometry in the
$x$-direction, Eqs. \eqref{be_rad} and \eqref{be_mat} take the following
form, for the first time step, dropping dependency notation (see Appendix
\ref{flcm}):
\begin{subequations}\label{time_start}
\begin{align}
\frac{1}{c}\frac{\partial I}{\partial t} + \mu\frac{\partial I}{\partial x}
+\sigma I =
  \frac{1-f_0}{2}\frac{\sigma_0b_0}{\sigma_p}\int_0^\infty\int_{-1}^1
  \sigma_0 I\ d\mu\ d\nu + &\sigma_0 f_0 2\pi B_0, \nonumber\\& %\hspace{40pt}
  0\leq x\leq\infty,\; |\mu|\leq 1,\; 0\leq t\leq t_1, \label{time_start_1}
\end{align}
\begin{equation}
\frac{c_{v0}}{\Delta_t}(T_1-T_0)=\frac{f_0}{\Delta_t}\int_0^{\Delta_t}
\int_0^\infty\int_ { -1 } ^1
  \sigma_0(I-2\pi B_0)\ d\mu\ d\nu\ dt.
\end{equation}

The initial and boundary conditions are given by:
\begin{equation}
I(0,\mu,\nu,t)=2\pi B_u\equiv2\pi B(\nu,T_u), \hspace{30pt} 
  0<\mu\leq1, \hspace{10pt} 0\leq t, \label{Bu_def}
\end{equation}
\begin{equation}
I(\infty,\mu,\nu,t)=2\pi B_0\equiv2\pi B(\nu,T_0), \hspace{30pt}
  -1\leq\mu\leq1,\hspace{10pt} 0\leq t,
\end{equation}
\begin{equation}
 I(x,\mu,\nu,0)=I_i, \hspace{30pt}0\leq x\leq\infty, |\mu|<1 \label{initcond}.
\end{equation}
\end{subequations}
Eq. \eqref{initcond} is the first deviation from \cite{WolLarDen}, where
the initial intensity was set to the same properties as the right boundary
condition, which assumes complete thermal equilibrium.


\subsection{Space}
As is necessary in most computational transport solutions, the continuous
properties governed by Eqs.\hspace{3pt}\eqref{time_start} must be spatially
discretized
into $J$ distinct
portions, or grid cells, each with a left boundary at $x_j$ and having width
$\Delta_{x,j}=x_{j+1}-x_j$.  The average temperature within each cell is given
by a
temperature averaging function:
\[T_j=\frac{1}{\Delta_{xj}}\int_{x_j}^{x_{j+1}}T(x)dx, \hspace{15pt}
  0\leq j\leq J-1. \]
In this manner, the discontinuous temperature approximation function $\tilde
T(x)$ approximates the physical continuous temperature function $T(x)$ as
follows:
\[\tilde T(x)\equiv\sum_{j=0}^{J-1}\chi_j(x)T_j\approx T(x),\]
where $\chi_j$ is a piecewise distribution, as follows:
\begin{equation}
\chi_j(x)\equiv
\begin{cases}
1 & x_j\leq x\leq x_{j+1}, \\
0 & \mbox{otherwise}.
\end{cases}
\end{equation}
Thus, temperature is often treated in as piecewise constant by
using $\tilde T(x)$ instead of $T(x)$ for explicit terms in Eqs.
\eqref{time_start} such as $\sigma$ and $B$.  The material equation then
appears as follows:
\begin{equation}
\frac{\tilde c_{v0}}{\Delta_t}(T_{j,1}-T_{j,0}) + 
  \tilde f_0 \tilde \sigma_p acT^4_{j,0} = 
  \frac{\tilde f_0}{\Delta_{x,j}
  \Delta_t}\int_0^\infty\tilde\sigma_0\int_{x_j}^{x_{j+1}}
\int_{\Delta_t}\int_{-1}^1 I \ d\mu \ dt\  dx\  d\nu.
\end{equation}

\subsection{Time, Part II}
The previous definitions, time-explicit coefficients, and grid data are typical
for the IMC equations.  In the following sections, we introduce additional
approximations in order to estimate the radiation energy deposited in the first
cell.
We start by assuming separability of the radiative
intensity $I$ into the collided ($\breve I$) and uncollided ($\hat I$)
parts so that $\hat I + \breve I=I$.

\subsubsection{Uncollided Intensity}
We approach the uncollided intensity first
because it is more analytically tractable and appears in the collided intensity
equation as a source term. The
following equations govern uncollided intensity:
\begin{subequations} \label{uncollided_1}
\begin{equation}
\frac{1}{c}\frac{\partial\hat I}{\partial t} + \mu\frac{\partial\hat
  I}{\partial x} + \sigma_0\hat I = 0,
\end{equation}
\begin{align}
\hat I(0,\mu,\nu,t)&=2\pi B_u,\hspace{15pt} 0<\mu\leq1, \hspace{10pt}0\leq t,\\
\hat I(\infty,\mu,\nu,t)&=0,\hspace{15pt} -1\leq\mu<0,\hspace{10pt}0\leq t,\\
\hat I(x,\mu.\nu,0)&=0, \hspace{15pt}0\leq x\leq\infty,\hspace{10pt}|\mu|\leq1.
\end{align}
\end{subequations}
Given the lack of any source terms in Eqs. \eqref{uncollided_1}, the
system is completely absorbing in nature.  By way of note, it could be solved
analytically, but will be treated by discretizing in time implicitly:
\[ \frac{1}{c}\left(\frac{\hat I_1-\hat I_0}{\Delta_t}\right) +
  \mu\frac{\partial}{\partial x}\hat I_1 + \sigma_0\hat I_1=0,\]
where $\hat I_k\equiv\hat I(t_k)$ designates the uncollided radiation intensity
at discrete time step $t_k$.  Because of the specified initial condition (i.e.,
all intensity existing at the start of the problem is collided
intensity, the uncollided intensity is zero throughout the mesh at time
$t=0$),
\[\frac{1}{c\Delta_t}{\hat I_1} + \mu\frac{\partial}{\partial x}\hat
I_1 + \sigma_0\hat I_1=0,\]
or, in operator form,
\begin{equation}
\mu\frac{\partial\hat I_1}{\partial x} +
  \left(\sigma_0+\frac{1}{c\Delta_t}\right)\hat I_1 = 0 \label{uncol_ode}.
\end{equation}
For $\mu\geq0$ the solution for the time-implicit uncollided radiation is found
using the integrating factor $e^{\sigma_0x+\frac{1}{c\Delta_t}x}\equiv
e^{\hat\Sigma(\nu)x}$:
\begin{equation}
\hat I_1(x,\mu,\nu)=2\pi B_ue^{-\hat\Sigma x/\mu} \label{uncol_solve}.
\end{equation}
This is limited to particles traveling in angles of positive $\mu$.  This is
practical for a problem with isotropically impinging radiation on once face
of a cell; however, further treatment is necessary in order to
account for significant influences on multiple faces of a cell.  This will be
addressed in Section \ref{propagate} for both single- and multidimensional
cases.

Taking the zeroth angular moment of Eq. \eqref{uncol_solve},
\[\hat\phi_1(x,\nu)\equiv\int_{-1}^1\hat I_1d\mu =
  \int_0^1 2\pi B_ue^{-\hat\Sigma x/\mu}d\mu,\]
\begin{equation} \label{uncol_mom1}
\hat\phi_1(x,\nu)=2\pi B_uE_2(\hat\Sigma x).
\end{equation}
where $E_n$ is the exponential integral defined as follows:
\[E_n(x)\equiv\int_0^1\mu^{n-2}e^{-x/\mu}d\mu,\hspace{15pt}0<x.\]

Eq.\ \eqref{uncol_mom1}, is the approximate solution of the
uncollided intensity.  Note that this derivation coincides precisely with the
process employed by Wollaber, Larsen, and Densmore \cite{WolLarDen}.

\subsubsection{Collided Intensity}
It is in this section that the deviation from previous derivation of the
discrete maximum principle occurs.  Because the collided intensity contains
all the source terms, it is more difficult to treat.  Eqs.
\eqref{collided} are the governing equations of the collided intensity:
\begin{subequations}\label{collided}
\begin{equation}
\frac{1}{c}\frac{\partial\breve I}{\partial t} + 
  \mu\frac{\partial\breve I}{\partial x} + \sigma_0\breve I =
  \frac{1-f_0}{2}\frac{\sigma_0b_0}{\sigma_p}\int_0^\infty\int_{-1}^1
    \sigma_0(\hat I+\breve I)d\mu d\nu +
  2\pi\sigma_0f_0B_0,
\end{equation}
\begin{align}
\breve I(0,\mu,\nu,t)&=0,\hspace{15pt} 0<\mu\leq1, \hspace{10pt}0\leq t,\\
\breve I(\infty,\mu,\nu,t)&=2\pi B_R,\hspace{15pt}
  -1\leq\mu<0,\hspace{10pt}0\leq t,\\
\breve I(x,\mu,\nu,0)&=I_i(x,\mu,\nu), \hspace{15pt}0\leq
  x\leq\infty,\hspace{10pt}|\mu|\leq1 \label{initcond2}.
\end{align}
\end{subequations}
Note that Eq.\ \eqref{initcond2} is the general initial condition employed
particular to this derivation.

To solve Eqs.\ \eqref{collided}, a standard diffusion approximation is
made, leading to the following:
\begin{equation}\label{diff_appr}
\frac{1}{c}\frac{\partial\breve\phi}{\partial t} -
  \frac{1}{3\sigma_0}\frac{\partial^2\breve\phi}{\partial x^2} +
  \sigma_0\breve\phi =
  (1-f_0)\frac{\sigma_0b_0}{\sigma_p}
    \int_0^\infty\sigma_0(\hat\phi+\breve\phi)d\nu +
  4\pi f_0\sigma_0B_0,
\end{equation}
where the scalar intensity $\breve\phi$ is
\[ \breve\phi(x,\nu,t)\equiv\int_{-1}^1\breve I(x,\mu,\nu,t)d\mu.\]
As in \cite{WolLarDen}, it is assumed that $\breve\phi$ has a frequency shape
given
by the Planckian spectrum $b_0$ for cell temperature $T_0$, allowing it to be
separated as follows:
\[\breve\phi(x,\nu,t)\approx b_0(\nu)\check\phi(x,t).\]
Defining the diffusion coefficient
\begin{equation}
D\equiv\int_0^\infty\frac{b_0(\nu)}{3\sigma_0(\nu)}d\nu,\label{defD}
\end{equation}
Eq. \eqref{diff_appr} can be integrated over all frequencies $\nu$ to
obtain
\begin{equation} \label{diff2}
\frac{1}{c}\frac{\partial\check\phi}{\partial t}
  -D\frac{\partial^2\check\phi}{\partial x^2} + f_0\sigma_p\check\phi
  =(1-f_0)\int_0^\infty\sigma_0\hat\phi d\nu + f_0\sigma_pacT^4_0.
\end{equation}
At this point, an implicit time discretization is
employed, although the approach is more general here given the undefined
initial condition throughout the mesh given in Eqs.\ \eqref{collided}. 
Implicit time discretization is employed as follows for any time-dependent
function $f(t)$:
\[ \bar f(t) \equiv\frac{1}{\Delta_t}\int_{t_n}^{t_{n+1}}f(t) dt 
  \approx f(t_{n+1}).\]
Additionally, we define the radiation temperature at time $t=0$ to satisfy:
\[ \int_0^\infty\check\phi_0\ d\nu\equiv\int_0^\infty\int_{-1}^1\check I_i\
d\mu\ d\nu \equiv acT_R^4,\]
where we define $T_R$ as the radiation field temperature within the cell at
time $t=0$.
Applying time differencing to Eq. \eqref{diff2}, we arrive at the
following:
\begin{equation}
\frac{\partial^2 \check\phi_1}{\partial x^2} - \lambda^2\check\phi_1 =
  \frac{1-f_0}{-D}\int_0^\infty \sigma_0\hat\phi d\nu -
  \frac{acT_R^4}{c\Delta_tD} - \frac{f_0\sigma_p}{D}acT_0^4,
\end{equation}
where $\lambda$ has been
introduced as
\[ \lambda^2\equiv\frac{f_0\sigma_p+1/c\Delta_t}{D}.\]

Using Eq. \eqref{uncol_mom1} in Eq. \eqref{diff2},
\begin{subequations}\label{ode}
\begin{equation}
\frac{\partial^2 \check\phi_1}{\partial x^2} - \lambda^2\check\phi_1 =
  \frac{1-f_0}{-D}\int_0^\infty\sigma_02\pi B_uE_2(\hat\Sigma x) d\nu -
  \frac{acT_R^4}{c\Delta_tD} - \frac{f_0\sigma_p}{D}acT_0^4. \label{ode_main}
\end{equation}
As with \cite{WolLarDen}, a Marshak boundary condition is set at the left:
\begin{equation}
\check\phi_1(0)-2D\frac{d\check\phi_1}{dx}\bigg|_{x=0}=0, \label{marshakBC}
\end{equation}
and the right boundary is
\begin{equation}
\lim_{x\to\infty}\check\phi_1(x)=acT_R^4. \label{infBC}
\end{equation}
To bring this derivation back in harmony with \cite{WolLarDen}, $\tilde A$ is
defined
\begin{equation}
\tilde A\equiv - \frac{acT_R^4}{c\Delta_tD} - \frac{f_0\sigma_p}{D}acT_0^4,
\end{equation}
and operator $L$
\begin{equation}
L(\xi)\equiv\frac{1-f_0}{-D}\int_0^\infty\sigma_02\pi B_u\xi d\nu,\label{defL}
\end{equation}
making Eq.\ \eqref{ode_simple}, the simplification of Eq.\ \eqref{ode_main},
identical in
appearance to Eq. (17g) in \cite{WolLarDen}:
\begin{equation}\label{ode_simple}
\frac{\partial^2\check\phi_1}{\partial x^2}-\lambda^2\check\phi_1 =
  \tilde A + L\left(E_2(\hat\Sigma x)\right).
\end{equation}
\end{subequations}
\\
We see by inspection that the remainder of the derivation in \cite{WolLarDen}
applies perfectly to
\eqref{ode_simple}, only trivially replacing $A$ everywhere with $\tilde A$. 
The final results arrived at in \cite{WolLarDen} are shown here, noting only
that $\tilde R$ (the average energy deposited in the first cell) and $\tilde A$
contain the radiation temperature $T_R$ as well as the material temperature
$T_0$.
\begin{align}\label{R}
\tilde R(\Delta_x,\Delta_t)&\equiv\frac{f_0\Delta_t2\pi}{\Delta_x}
  \int_0^\infty\frac{\sigma_0B_u}{\hat\Sigma}\left(\frac{1}{2}-
    E_3(\hat\Sigma\Delta_x)\right)d\nu \nonumber\\
  &+\frac{c_1f_0\Delta_t\sigma_p}{\lambda\Delta_x}(1-e^{-\lambda\Delta_x}) +
  \frac{f_0\Delta_t\sigma_p}{\lambda^2\Delta_x}\left[-\tilde A\Delta_x
    -L\left\{\frac{1}{\hat\Sigma}\left(\frac{1}{2}
    -E_3(\hat\Sigma\Delta_x)\right)\right\}\right] \\
  &\hspace{15pt}+\frac{f_0\Delta_t\sigma_p}{\Delta_x\lambda^4}L\Bigg\{\hat\Sigma
    \bigg[e^{-\lambda\Delta_x}\mbox{Ei}[(\lambda-\hat\Sigma)\Delta_x] +
    e^{\lambda\Delta_x}\mbox{Ei}[-(\lambda+\hat\Sigma)\Delta_x] \nonumber \\
  &\hspace{140pt} -2\mbox{Ei}(-\hat\Sigma\Delta_x)+ \nonumber
    \ln\frac{\hat\Sigma^2}{(\lambda+\hat\Sigma)|\lambda-\hat\Sigma|}
    \bigg]\Bigg\},
\end{align}
Thus, the IMC equations can be expected to satisfy the maximum principle if the
following inequality is satisfied, calculating $\tilde R$ as in Eq.\ \eqref{R}.
\begin{equation}\label{DMP}
\Delta_t<\frac{c_v(T_u-T_0)}{\tilde
R(\Delta_x,\Delta_t)-f_0c\sigma_paT_0^4}.
\end{equation}
