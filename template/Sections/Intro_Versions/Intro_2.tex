%=================%
% Section:        %
%    Introduction %
%=================%

\ResetSingleSpace

\begin{center}
{\textbf{
THE k-EIGENVALUE FROM MULTI-ENERGY GROUP NEUTRON TRANSPORT
IN BINARY STOCHASTIC MEDIA IN PLANAR GEOMETRY
}}
\end{center}

\ResetDoubleSpace

\vskip0.25in

\begin{center}
\section{INTRODUCTION}
\label{sec:Intro}
\end{center}

\setcounter{page}{1}
\thispagestyle{empty}

\vskip-0.1in

\noindent
	\indent In recent years, a relatively new field of study has emerged in the radiation transport
	community focused on the interaction of particles moving through a stochastic or random
	heterogeneous background medium.  The background medium is stochastic or random for 
	this class of problems since the precise location of any one of the constituents
	of the medium is unknown.  In most Nuclear Engineering applications, the 
	background media has a known location and is fixed.  Very accurate modeling
	practices have been developed for particle transport and diffusion calculations by dividing
	a heterogeneous background media of this type into homogenous sub-sections.  These
	sub-sections are chosen to be those parts of the medium where the composition is (or is  
	nearly) homogenous, or the microstructure of the composition has a size less than a mean
	free path of the particle, and can be homogenized with an acceptable loss of accuracy.

\noindent
	\indent The common practice of homogenizing a heterogeneous medium has 
	various effects on modeling the interaction of the particle with the medium.  As a particle
	traverses a
	homogenous medium, it encounters all of the constituents of the medium
	with equal probability during its lifetime.  In reality, this might not be the case.  The
	particle may encounter pockets of distinctly different material in the microstructure of the
	medium, causing profoundly different bulk particle flow effects at the macroscopic level
	than would be predicted by a homogeneous model.  In these cases, the microstructure
	has to be given specific treatment.
	
\noindent
	\indent A large body of work now exists for modeling particle transport in stochastic media,
	but this area of research is still far from mature.  According to reference~\cite{Lew:93} and
	reference~\cite{Lew:93}  the original interest in these problems was generated to accurately
	describe radiative transfer in Rayleigh-Taylor mixed regions of immiscible fluids found in a
	capsule undergoing the implosion, and subsequent explosion, in inertial confinement fusion.
	This is a system where there is radiative transfer through a medium of drastically
	different densities at the turbulent propagation front of the shock wave, and homogenization
	produces solutions with unacceptable error.  Since
	then, focus has been turned to other problems spanning a diverse set of disciplines interested
	in particle transport, where a stochastic treatment of the background medium is necessary.
	The above example extends to larger scale modeling of astrophysical phenomena,
	such as the interior of stars.  Modeling properties of the microstructure is also necessary in
	global climate modeling, where radiation transport must be preformed through
	heterogeneous clouds, or in oceanography models of radiative transfer through murky
	ocean, or ocean layers with vastly different densities due to temperature and pressure.
	Explicitly modeling heterogeneities may be necessary in some nuclear reactor applications
	such as the thermohydraulic and neutron density coupling in boiling water reactors where the
	neutrons traverse a bubbly flow (water-void mixtures). Pebble bed reactors have at least three
	levels of stochasticity.  The fuel pebble itself is a random mixture of smaller fuel spheres in a
	graphite matrix used for moderation.  These pebbles are tightly packed and randomly mixed
	in the core region surrounded by gaseous coolant.  An additional layer of stochasticity is
	introduced after the reactor has been in operation for some time and the fuel content is
	depleted in the pebbles.  Common materials used in radiation shielding may have to be
	treated as stochastic media, such as cement, where particles may interact with large regions
	of constituents with very different neutron interaction probabilities.

\noindent 
	\indent All materials are heterogeneous in nature, and the size of a single component 
	segment, referred to as a transition length or chord length, of a heterogeneous material can
	be variable.  Any given chord length of any material in a composition depends on some
	measured (or assumed known) statistical distribution.  Each material chord length may 
	depend on different
	statistical distributions.  The complete set of statistical distribution functions describing the
	chord lengths of each material in the medium, are the governing statistics of how the
	microscopic, heterogeneous materials are randomly mixed to compose the macroscopic
	medium. Since the location of the materials composing the medium are only known in this
	statistical sense, then the particle flux in such a material can only be described as a function
	of space with statistical moments, such as the mean flux and variance.

\noindent
	\indent Once the mixing statistics describing how the individual materials are mixed in the
	medium are measured or assumed, one can begin developing strictly deterministic models
	for the statistical system.  A major effort has been to derive computationally inexpensive
	deterministic models for the ensemble average flux, and higher order statistical moments
	such as the variance of the flux, in a stochastic medium.  Most of the work to date has been
	in the development of models for the particle flux in a medium composed of two or more
	components, whose material chord lengths are assumed to be well represented by a
	Markovian distribution.  The primary focus of this work has been for problems of the
	boundary value variety with Dirichlet boundary conditions.
	
\noindent
	\indent Boundary value problems of the form $\underline{\underline{\mathbf{A}}}
	\underline{\psi}=\underline{\mathbf{q}}$ are not the only major class of problem that can be
	investigated that are of interest in the Nuclear Engineering community.  Another class of
	problem that is of great interest is the eigenproblem of the form
	$\underline{\underline{\mathbf{A}}}\underline{\psi}=\lambda\underline{\psi}$, where the
	eigenvalue, $\lambda$, gives and indication of the criticality of the system, and the resulting
	flux is the eigenvector, which is not unique.  When considering the criticality of the system, 
	one can evaluate the criticality as a function of time or in the limit of �very long� reactor
	operation times.  In the first case the nuclear engineer may be considering a problem
	where a given
	reactor is operating steadily at some initial power, when at some later time the reactor
	undergoes some change in the reactivity, causing a spike in the reactor power.  This can
	occur during normal operation through the movement of control rods, neutron �poisons�
	added to the reactor chemistry, or burnable poisons placed in the fuel for control purposes.
	The buildup and decay of fission products, like Iodine and Xenon or Promethium and
	Samarium, is common during startup and shutdown which adds or removes reactivity.
	Accident scenarios such as loss of coolant or loss of pressure can cause reactivity
	fluctuations for which robust safety systems have been designed in all modern nuclear
	reactors.  This type of reactor criticality scenario in the context of a stochastic background
	medium, is not the subject of this thesis, although some of these problems, such as
	burnable poisons with somewhat random location in the fuel, are excellent candidates for
	potential stochastic media transport research.
	
\noindent
	\indent The stochastic eigenproblem considered in this thesis are those problems where
	there are given distributions of fuel and moderator loading for the reactor, and the criticality
	is evaluated in the limit of �very long� reactor times without a change in this material
	distribution.  To further
	understand this problem it is helpful to first explore the Boltzmann transport
	equation, followed by a simple conceptual example in transport theory, leading to an
	understanding of the eigenproblem for criticality (adopted from reference ~\cite{Lew:93}).  A
	complete  description of the distribution of neutrons in any medium is given by the
	Boltzmann transport equation (~\cite{Lew:93}),
	\begin{equation}
		\frac{\partial{n}}{\partial{t}}+\emph{v}\mathbf{\hat{\Omega}} \!\cdot\! \mathbf{\nabla}{n}
		 + \emph{v}\sigma{n} = \int_{4\pi} {d}\mathbf{\hat{\Omega}}^{'}\int_{0}^{\infty}{d}{E}^{'}
		 \emph{v}^{'}\sigma_{s}({E}^{'}\rightarrow{E},\mathbf{\hat{\Omega}}^{'}\rightarrow
		 \mathbf{\hat{\Omega}}){n}^{'}+q
	\label{eq:transport1}
	\end{equation}
	where, \vspace{10pt} \\
	\begin{tabular}{ll}
		${t}$ \hfill = & time,  \vspace{10pt} \\
		${E}$ \hfill = & energy, \vspace{10pt} \\
		${E}^{'}$ \hfill = & some energy other than E, \vspace{10pt} \\
		$\mathbf{v}$ \hfill = & neutron velocity vector,  \vspace{10pt} \\
		$\emph{v}$ \hfill = & ${|\mathbf{v}|}$; neutron speed, \vspace{10pt} \\
		$\emph{v}^{'}$ \hfill = & some neutron speed other than $\emph{v}$ , \vspace{10pt} \\
		$\mathbf{\hat{\Omega}}$ \hfill = & $\frac{\mathbf{v}}{|\mathbf{v}|}$; unit vector 
			for neutron direction of travel \\ & with both a  polar and azimuthal component,
			\vspace{10pt} \\
		$\mathbf{\hat{\Omega}} ^{'}$ \hfill = & some direction of travel other than
			$\mathbf{\hat{\Omega}}$, \vspace{10pt} \\
		$\sigma$ \hfill = & $\sigma(\mathbf{r},E)$; total cross-section, or the probability \\ &
			that a neutron at point $\mathbf{r}$ with energy $E$ will \\& undergo some
			interaction, \vspace{10pt} \\
		$\sigma_s({E}^{'}\rightarrow{E},\mathbf{\hat{\Omega}}^{'}
			\rightarrow\mathbf{\hat{\Omega}})$ \hfill = & $\sigma_s(\mathbf{r},{E}^{'}
			\rightarrow{E}, \mathbf{\hat{\Omega}}^{'} \rightarrow\mathbf{\hat{\Omega}})$;
			scattering cross-section, \\ & or the probability that a neutron at
			point $\mathbf{r}$ with \\& energy ${E}^{'}$ traveling in direction
			$\mathbf{\hat{\Omega}}^{'}$ will scatter \\ & into the energy and direction of
			interest, ${E}$,  $\mathbf{\hat{\Omega}}$, \vspace{10pt} \\
		${n}={n(\mathbf{r},E,\mathbf{\hat{\Omega}},t)}$ \hfill = & neutron density at point
		$\mathbf{r}$, with energy ${E}$, moving \\ & in direction $\mathbf{\hat{\Omega}}$,  at
		time ${t}$, \vspace{10pt} \\
		${q}={q(\mathbf{r},E,\mathbf{\hat{\Omega}},t)}$ \hfill = & source of neutrons at point
		$\mathbf{r}$, producing neutrons \\ & with energy ${E}$, moving in direction
		$\mathbf{\hat{\Omega}}$,  at time ${t}$. \vspace{10pt} \\
	\end{tabular} \vspace{10pt} \\

\noindent
	\indent Thus, the Boltzmann transport equation is an exact equation for the angular neutron
	density through a balance of gain and loss mechanisms of neutrons in an arbitrary
	phase space, ${(\mathbf{r},E,\mathbf{\hat{\Omega}},t)}$. Possible gain mechanisms, or
	ways in which neutrons appear in ${(\mathbf{r},E,\mathbf{\hat{\Omega}},t)}$ include:

\begin{itemize}
	\item \textbf{Sources} into ${(\mathbf{r},E,\mathbf{\hat{\Omega}},t)}$ - fission or nuclide decay
		resulting in a neutron
	\item \textbf{Streaming} into ${(\mathbf{r},E,\mathbf{\hat{\Omega}},t)}$ - neutrons moving into
	${(\mathbf{r},E,\mathbf{\hat{\Omega}},t)}$ from ${(\mathbf{r}^{'},E,\mathbf{\hat{\Omega}},t)}$
	\item \textbf{Collision} in ${(\mathbf{r},E,\mathbf{\hat{\Omega}},t)}$ - neutrons in
	${(\mathbf{r},{E}^{'},\mathbf{\hat{\Omega}}^{'},t)}$ colliding into
	${(\mathbf{r},E,\mathbf{\hat{\Omega}},t)}$
		
	are balanced with possible loss mechanisms, where neutrons disappear from 
	${(\mathbf{r},E,\mathbf{\hat{\Omega}},t)}$ by:
	
	\item \textbf{Leakage} out of ${(\mathbf{r},E,\mathbf{\hat{\Omega}},t)}$ - neutrons from 
	${(\mathbf{r},E,\mathbf{\hat{\Omega}},t)}$ into ${(\mathbf{r}^{'},E,\mathbf{\hat{\Omega}},t)}$
	\item \textbf{Collision} out of ${(\mathbf{r},E,\mathbf{\hat{\Omega}},t)}$ - neutrons in
	${(\mathbf{r},E,\mathbf{\hat{\Omega}},t)}$ colliding into
	${(\mathbf{r},{E}^{'},\mathbf{\hat{\Omega}}^{'},t)}$ or neutrons absorbed (including fission).	
\end{itemize}
\noindent
	The Boltzmann transport equation is a linear, integrodifferential equation for 
	the angular neutron density in a seven dimensional phase space,
	($\mathbf{r}=x,y,z;{E};\mathbf{\hat{\Omega}}=\Theta,\gamma;t)$.

\noindent
	\indent Defining the angular neutron flux as a product of the angular neutron density and
	the neutron speed, or,
	\begin{equation}
		\psi(\mathbf{r},E,\mathbf{\hat{\Omega}},t) = \emph{v} {n}
			(\mathbf{r},E,\mathbf{\hat{\Omega}},t),
	\end{equation}
	 it is convenient to rewrite the neutron transport equation in terms of the angular neutron flux,
	 \begin{equation}
		\frac{\partial{\psi}}{\partial{t}}+\mathbf{\hat{\Omega}} \!\cdot\! \mathbf{\nabla}{\psi}
		 + \sigma{\psi} = \int_{4\pi} {d}\mathbf{\hat{\Omega}}^{'}\int_{0}^{\infty}{d}{E}^{'}
		\sigma_{s}({E}^{'}\rightarrow{E},\mathbf{\hat{\Omega}}^{'}\rightarrow
		\mathbf{\hat{\Omega}}) \psi^{'}+q,
	\end{equation}
	 since the neutron scalar flux, defined as, 
	 \begin{equation}
		\phi(\mathbf{r},t) = \int_{4\pi} {d}\mathbf{\hat{\Omega}}^{'}\int_{0}^{\infty}{d}{E}^{'}
		 \psi^{'}(\mathbf{r},{E}^{'},\mathbf{\hat{\Omega}}^{'},t)
	\end{equation}
	is often the quantity of interest as it is the simplest to conceptualize and used in
	calculating reaction rates.  
	
\noindent
	\indent With the Boltzmann transport equation defined (Eq.~(\ref{eq:transport1})), we will 
	use it to define the eigenvalue problem we intend to investigate for a stochastic multiplying
	background medium.  Consider the transport problem governed by Eq.~(\ref{eq:transport1})
	where the first generation of neutrons is a pulsed neutron source given as $q_1$
	resulting in the angular neutron density of the first generation of neutrons in a multiplying
	medium, 
\begin{equation}
		\frac{\partial{n_1}}{\partial{t}}+\emph{v}\mathbf{\hat{\Omega}} \!\cdot\! \mathbf{\nabla}{n_1}
		 + \emph{v}\sigma{n_1} = \int_{4\pi} {d}\mathbf{\hat{\Omega}}^{'}\int_{0}^{\infty}{d}{E}^{'}
		 \emph{v}^{'}\sigma_{s}({E}^{'}\rightarrow{E},\mathbf{\hat{\Omega}}^{'}
		 \rightarrow\mathbf{\hat{\Omega}}){n_1}^{'}+q_1.
\end{equation}
\noindent
	Integrating over all time ($0\le{t}\le{\infty}$) results in the first term on the left hand side going to
	zero, since the pulsed source is of finite duration.  Those pulsed neutrons in that generation either
	scatter or leak out of the system, or are absorbed in the multiplying medium in either fuel or other
	material.  We shall denote the time integrated neutron source as 
	$\tilde{q_1}=\tilde{q_1}(\mathbf{r},E,\mathbf{\hat{\Omega}})$ and the time integrated angular
	neutron density as $\tilde{n_1}=\tilde{n_1}(\mathbf{r},E,\mathbf{\hat{\Omega}})$ giving,
\begin{equation}
	\emph{v}\mathbf{\hat{\Omega}} \!\cdot\! \mathbf{\nabla}\tilde{n_1}
		 + \emph{v}\sigma{\tilde{n_1}} = \int_{4\pi} {d}\mathbf{\hat{\Omega}}^{'}
		 \int_{0}^{\infty}{d}{E}^{'}\emph{v}^{'}\sigma_{s}({E}^{'}\rightarrow{E},
		 \mathbf{\hat{\Omega}}^{'}\rightarrow \mathbf{\hat{\Omega}})\tilde{n_1}^{'}+\tilde{q_1}.
\end{equation}

\noindent
	The second generation of neutrons, $\tilde{n_2}$, are produced by those neutrons in
	$\tilde{n_1}$ which are absorbed in fuel and cause a fission.  Thus, fission is the
	process by which a generation of neutrons gives birth to a subsequent generation, making
	it the event separating generations.  The source, $\tilde{q_2}$,
	producing the generation, $\tilde{n_2}$, is given by,
\begin{equation}
	\tilde{q_2} = \chi({E})\int_{4\pi} {d}\mathbf{\hat{\Omega}}^{'}\int_{0}^{\infty}{d}{E}^{'}
		 \nu{\sigma_f} \emph{v}^{'}\tilde{n_1}^{'}
	\label{eq:src-gen2}
\end{equation}
	where, \vspace{10pt} \\
	\begin{tabular}{ll}
		${\chi}$ \hfill = & ${\chi({E})}$; the probability that a neutron produced from fission
			will \\ & have energy ${E}$, where, ${\int_{0}^{\infty}{\chi}({E})d{E}=1}$,
			\vspace{10pt} \\	
		${\nu}$ \hfill = & ${\nu({E}^{'})}$; the mean number of neutrons produced by fission
		from \\ & a neutron with energy ${E}^{'}$\vspace{10pt} \\
		${\sigma_f}$ \hfill = & ${\sigma_f(\mathbf{r},{E}^{'})}$; fission cross section, or the
		probability that a neutron \\ & with energy ${E}^{'}$ is absorbed at point $\mathbf{r}$
		in fuel will fission.  \vspace{10pt} \\
	\end{tabular} \vspace{10pt} \\
	
\noindent
	Equations can be written in this manner for the ${\emph{(i)}^{th}}$ generation of neutrons
	produced by the ${\emph{(i-1)}^{th}}$ generation or by the following recursive equation,
\begin{multline}
	\emph{v}\mathbf{\hat{\Omega}} \!\cdot\! \mathbf{\nabla}\tilde{n_\emph{i}}
		 + \emph{v}\sigma{\tilde{n_\emph{i}}} = \int_{4\pi} {d}\mathbf{\hat{\Omega}}^{'}
		 \int_{0}^{\infty}{d}{E}^{'}\emph{v}^{'}\sigma_{s}({E}^{'}\rightarrow{E},
		 \mathbf{\hat{\Omega}}^{'}\rightarrow \mathbf{\hat{\Omega}})
		 \tilde{n_\emph{i}}^{'} + \\
		 \chi({E})\int_{4\pi} {d}\mathbf{\hat{\Omega}}^{'}\int_{0}^{\infty}{d}{E}^{'}
		 \nu{\sigma_f} \emph{v}^{'}{\tilde{n}_{i-1}}^{'}.
	\label{eq:recurs}
\end{multline}

\noindent
	As the number of generations becomes large for Eq.~(\ref{eq:recurs}), the ratio of
	successive generations will be a constant, or, 
\begin{equation}
	\lim_{i\rightarrow \infty}\frac{\tilde{n_i}}{\tilde{n}_{i-1}}=constant={k}
\end{equation}
\noindent
	\indent This constant ${k}$ is known as the multiplication factor and indicates the
	"criticality" of the system.  Calculating the multiplication factor is most
	often approached as an eigenproblem, where ${\lambda=\frac{1}{k}}$ is the eigenvalue
	for the system and ${\psi}$, the angular neutron flux, is the eigenvector.  In practical
	application, the ${\nu}$ of Eq.~(\ref{eq:src-gen2}) is replaced by ${\nu/{k}}$ , or
	${\lambda{\nu}}$, adjusting the average number of neutrons per fission to balance the
	equation.  The maximum ${\lambda}$ (or minimum ${k}$) and resulting 
	eigenvector (fundamental mode) are the quantities of interest.  For a system where
	${k} < {1}$ the neutron density in successive generations is decreasing, or the number
	of neutrons per fission, ${\nu/{k}}$, required to make the system exactly critical is larger
	than ${\nu}$.  This is known as a subcritical system.  If ${k} > 1$ the neutron density in
	successive generations is increasing,  or the number of neutrons per fission, ${\nu/{k}}$,
	required to make the system exactly critical is smaller than ${\nu}$.  This is known as a
	supercritical system.  For ${k} = 1$, the neutron density in successive generations is
	unchanging, or the number of neutrons produced in fission, ${\nu}$, is exactly the same
	as the number of neutrons removed from the system in each generation.  This is known
	as a critical system.  The resulting time-independent governing equation for this
	eigenproblem, written here in terms of the angular flux,
\begin{multline}
	\mathbf{\hat{\Omega}} \!\cdot\! \mathbf{\nabla}{\psi}
		 + \sigma{\psi} = \int_{4\pi} {d}\mathbf{\hat{\Omega}}^{'}\int_{0}^{\infty}{d}{E}^{'}
		\sigma_{s}({E}^{'}\rightarrow{E},\mathbf{\hat{\Omega}}^{'}\rightarrow
		\mathbf{\hat{\Omega}})\psi^{'}+ \\ \frac{\chi({E})}{k} \int_{4\pi}{d}\mathbf
		{\hat{\Omega}}^{'}\int_{0}^{\infty}{d}{E}^{'}\nu{\sigma_f}{\psi}
	\label{eq:crit}
\end{multline}
\noindent is the subject of this thesis, applied to systems in which the multiplying
	background media is stochastic.  As the literature review in the next section will reveal,
	only very recently has research begun with the focus of investigating how criticality is
	affected by a stochastic multiplying background medium.  In addition to this, we are not
	aware of any published research regarding the criticality of a stochastic background
	medium in which the energy dependence of Eq.~(\ref{eq:crit}) is included.  Most of the
	research thus far has considered only single energy (one-speed) transport and
	diffusion problems.  

 