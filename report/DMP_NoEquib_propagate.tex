\subsection{Propagating DMP}\label{propagate}
An assumption throughout this paper thus far has been a Marshak boundary on the
left and a generally uninteresting initial and right boundary condition. 
This allowed us to assume only steep temperature gradients on one side ofthe
left boundary cellduring the first time step. We now desire to apply the DMP in
general to each cell in the mesh, so
that an inequality similar to Eq.\ \eqref{DMP} can be computed for a cell with
steep temperature gradients on
multiple sides.  We will do this by accounting for impingent radiation on
multiple faces of a cell, although we will continue to treat teh radiation as
being isotropic and Planckian.  The primary location of the previous assumption
allowed us to
find the solution in Eq.\ \eqref{uncol_solve}.  This was then used to calculate
the uncollided scalar intensity $\hat\phi_0$.  In the 1D case, $\hat\phi_0$ can
be found more accurately as follows:
\begin{equation}
\hat\phi_1\equiv\int_{-1}^1 \hat I_i\ d\mu = \int_{-1}^0 \hat I_i\ d\mu + 
  \int_0^1 \hat I_i\ d\mu. \label{uncol_gen}
\end{equation}
Because the initial conditions prescribe no uncollided intensity in a cell
initially, the intensity present at a point $x$ can only originate from incident
intensity on the boundaries.  The differential equation describing this is
given in Eq.\ \eqref{uncol_ode} and is solved in Eq.\ \eqref{uncol_solve}. 
There is physical meaning to the terms: $2\pi B_u$ represents the initial
spectrum of intensity found at the Marshak boundary entering the cell, and
$e^{-\Sigma x/\mu}$ attenuates this intensity through material interactions up
to point $x$.  Hence, a similar expression for uncollided flux arriving at $x$
from the right boundary can be derived as well.

Let $B_L$ replace $B_u$ as the Planck radiation spectrum on the left boundary,
and let $B_R$ represent a similar spectrum on the right boundary.  Similarly
let $\hat I_{L,i}$ and $\hat I_{R,i}$ represent the incident uncollided
intensity at the left and right right boundary, respectively.  Eq.\
\eqref{uncol_mom1} becomes
\begin{align}
\hat\phi_1&\equiv\int_{-1}^1 \hat I_i\ d\mu \nonumber ,\\
&=\int_{-1}^0 \hat I_{R,i}\ d\mu + 
  \int_0^1 \hat I_{L,i}\ d\mu \nonumber ,\\
&=\int_{-1}^0 \hat I_{R,i}\ d\mu +
  \int_0^1 2\pi B_Le^{-\hat\Sigma x/\mu}\ d\mu. \label{uncol_gen2}
\end{align}
Because of the arbitrary definition of positive $\mu$, we expect $\hat I_{R,i}$
to have a similar form to $\hat I_{L,i}$.  However, the attenuation term must
be treated differently for negative $\mu$ radiation.  Because the radiation
enters at the right and moves to the left, the attenuation distance for the
right side becomes $\Delta_x-x$. Thus by inspection and symmetry,
\begin{equation}
\int_{-1}^0 \hat I_{R,i}\ d\mu
  =\int_{-1}^0 2\pi B_R e^{-\hat\Sigma(\Delta_x-x)/\mu}\ d\mu.\label{IR1}
\end{equation}
To continue shaping this new first term to correlate with the second term,
consider that in the first integral all values of $\mu$ are negative.  Given
this condition,
\[\mu=-|\mu|, \hspace{30pt} \mu\leq0.\]
Substituting this into Eq.\ \eqref{IR1} and making a change of integration
variable to $|\mu|$, we obtain an expression very similar to the second term in
Eq.\ \eqref{uncol_gen2}:
\begin{equation}
\int_{-1}^0 \hat I_{R,i}\ d\mu = \int_0^1 2\pi B_R
  e^{-\hat\Sigma(x-\Delta_x)/|\mu|}\ d|\mu|.
\end{equation}
Since the new attenuation coefficient is identical in form to that in Eq.\
\eqref{uncol_solve}, the integral solution has the same form.  Thus,
\begin{equation}
\hat\phi_1=2\pi B_LE_2(\hat\Sigma x) +
  2\pi B_RE_2\left(\hat\Sigma(\Delta_x-x)\right) \label{phi1_LR}
\end{equation}
Eq.\ \eqref{phi1_LR} is sufficiently different from Eq.\ \eqref{uncol_solve}
that the original operator $L$ is divided into an operator $L_L$ for the left
side and $L_R$ for the right:
\begin{subequations}
\begin{align}\label{L_LR}
L_L(\xi)&\equiv\frac{1-f_0}{-D}\int_0^\infty\sigma_02\pi B_L\xi\ d\nu, \\
L_R(\xi)&\equiv\frac{1-f_0}{-D}\int_0^\infty\sigma_02\pi B_R\xi\ d\nu,
\end{align}
\end{subequations}
which, using the same procedure as in \cite{WolLarDen}, leads to a new
particular solution for $\breve\phi_1$:
\begin{align}
\breve\phi^p_1=\frac{1}{\lambda^2}\bigg[
  &-A-L_RE_2(\hat\Sigma(x-\Delta_x))+L_R\frac{\hat\Sigma}{2\lambda}\left(
    e^{-\lambda x}g(x-\Delta_x)+
    e^{\lambda x}E_1[(\lambda+\hat\Sigma)(\hat\Sigma(x-\Delta_x))]
    \right)\nonumber\\
  &-L_LE_2(\hat\Sigma x)+L_L\frac{\hat\Sigma}{2\lambda}\left(
    e^{-\lambda x}g(x)+e^{\lambda x}E_1[(\lambda+\hat\Sigma)x]\right)\bigg],
\end{align}
where
\begin{equation}
g(x)\equiv\int\frac{e^{(\lambda-\hat\Sigma)x}}{x}dx=
\begin{cases}
\mbox{Ei}[(\lambda-\hat\Sigma)x], & \lambda\neq\hat\Sigma, \\
\ln{x}, & \lambda=\hat\Sigma,
\end{cases}
\end{equation}
which makes use of the extension to Ei(x) noted in \cite{MathFunc}:
\[\mbox{Ei}(-x)\equiv-E_1(x),\; 0<x.\]

Next we apply the Marshak and finite boundary conditions.  The homogeneous
solution for $\breve\phi_1$ is the same as derived earlier, and we still set
$c_2=0$ to preserve the condition in Eq.\ \eqref{infBC}.  After some algebraic
manipulation, and noting the convenient equal divergence and limits described
in \cite{WolLarDen}, it can be shown that the average energy deposited in a
cell ($\tilde R$) follows the superposition principle for left and right sides
of the cell.  That is to say,
\[\tilde R = \tilde R_L + \tilde R_R.\]

The temperature update in a cell then becomes
\[\frac{c_v}{\Delta_t}(T_1-T_0)+f_0c\sigma_p aT_0^4=\tilde R_L+\tilde R_R,\]
\begin{equation}
T_1-T_0=\frac{\Delta_t}{c_v}(\tilde R_L+\tilde R_R-f_0c\sigma_p aT_0^4).
\label{tempUpd}
\end{equation}
Two separate equations fall out of Eq.\ \eqref{tempUpd}, one for each side of
the cell.  Since both the left boundary temperature $T_L$ and the right
boundary temperature $T_R$ limit the time step in similar ways, we here
define the maximum temperature
\[T_m\equiv\mbox{max}(T_L,T_R). \]
\begin{equation}\label{boundTempUpd}
T_m-T_1=T_m-T_0-\frac{\Delta_t}{c_v}(\tilde R_L+\tilde R_R-f_0c\sigma_p aT_0^4),
\end{equation}
As in \cite{WolLarDen}, the maximum principle is only satisfied if the boundary
temperature is greater than the update temperature, meaning the right side of
Eq.\ \eqref{boundTempUpd} is positive:
\begin{equation}\label{DMP_LR}
T_m-T_0>\frac{\Delta_t}{c_v}(\tilde R_L+\tilde R_R-f_0c\sigma_p aT_0^4),
\end{equation}
or, rearranging, the maximum principle can be stated in the same form as that
in \cite{WolLarDen},
\begin{equation}
\Delta_t < \frac{c_v(T_m-T_0)}{\tilde R_L(\Delta_x,\Delta_t)+\tilde
  R_R(\Delta_x,\Delta_t) - f_0c\sigma_p aT_0^4},
\end{equation}

It should be noted that the term $T_m-T_0$ in the numerator here is misleading. 
An analysis of the frequency-independent DMP leads to an inequality with
$T_m^4-T_0^4$ in the denominator, for example.

\subsection{Multidimensional Considerations}
Given the results achieved in the one-dimensional case, the maximum principle in
multiple dimensions can be extrapolated by adjusting
$\tilde R$.  Since the average energy deposited ($\tilde R$) is
separable in that it follows superposition rules, the composite term is as
follows:
\begin{equation}
\tilde R_{tot} = \sum_{s=1}^S \tilde R_s(\Delta_x,\Delta_t),
\end{equation}
where $S$ is the number of faces on the cell in question and $s$ represents one
of those faces.

The question of what to use for $\Delta_x$, the spatial discretization, in
higher dimensions is a valid one, and needs addressing.  In one dimension, it is
obvious that the cell size is equal to its length. However, in multiple
dimensions, the cell size is an area or volume, and linear distance
becomes more difficult to define uniquely.  To resolve this, we turn to the work
done by Bardsley and Dubi \cite{BardDub}, who expound the average-chord-length
theorem developed by Dirac \cite{Dirac}. This theorem states that for any
three-dimensional volume, the average chord length within that volume is given
by
\begin{equation}
\bar\Delta_x = 4\frac{V}{A},
\end{equation}
where $V$ is the volume and $A$ is the surface area of the volume.  We use this
definition to describe the value $\Delta_x$ in three dimensions to redefine the
discrete maximum inequality
\begin{equation}
\Delta_t<\frac{c_v(T_m-T_0)}{\tilde R_{tot}(\Delta_x,\Delta_t)-
  f_0c\sigma_paT_0^4},  \hspace{20pt} T_m=\mbox{max}(T_s).\label{FINAL dmp}
\end{equation}
