\section{Conclusions}
We undertook in this work to add to the theory surrounding the discrete
maximum principle as it applies to the implicit Monte Carlo equations, which in
turn govern radiative thermal transport.  First, we generalized previous
definitions of the principle \cite{WolLarDen} to allow for non-equilibrium
starting conditions between the radiation field and material temperatures. 
Next, the principle was extended to predict maxiumum principle violations in
conditions where strong temperature gradients due to isotropically impinging,
Planckian-distributed radiation fields on multiple faces of a
region of interest.  Lastly, approximations were applied in order to allow the
discrete maximum principle to be applied in use codes.%, and we demonstrated
%that such implementation produced maximum principles that predicted violations
%of physical behaviour in good coincidence with where those voilations actually
%occur in code.

There is still significant opportunity for future work on developing this
discrete maximum principle.  Assumptions such as the linear superposition of
radiation energy deposited in a cell from multiple boundaries deserve further
analysis to support such claims.  In addition, no effort has been made to
analyze the effect of a negative temperature gradient on unphysical behavior;
rather, it has been assumed that unphysical effects from such radical cooling
will be substantially less limiting in time step than heating effects.  Further
exploration of negative temperature gradients may help to lead to a more
general discrete maximum principle still.