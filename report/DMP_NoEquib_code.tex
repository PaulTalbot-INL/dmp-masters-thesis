\subsection{Coding DMP in \texttt{clubimc}}
The real purpose of the discrete maximum principle is to determine when a
selection of time step and spatial discretization is likely to produce
unphysical results in an IMC code.  To demonstrate the development of the DMP
up to this point, functionality was added to Los Alamos National
Labratory's \texttt{clubimc} C++ library, which provides a host of tools for
solving the IMC equations using Monte Carlo techniques.

In order to implement this code, two classes were created.  The first,
\texttt{DMP\_Helper}, serves to gather the information required to calculate the
inequality that governs the DMP, such as material properties, local
temperatures, neighboring temperatures, fleck factors, opacities, and the
spatial and discretization step, as well as the multigroup or grey frequency
information.  The second class, \texttt{DiscreteMaxPrinciple}, performs and
retains calculation of the inequality value as well as a boolean pass/fail for
each cell.  A further description of each class is included here.

\subsubsection{\texttt{DMP\_Helper}}
The purpose of this class is to gather the necessary information to calculate
the discrete maximum principle for each cell on the mesh.  In its current
incarnation, it follows the following process:
\begin{itemize}
 \item Dependencies within \texttt{clubimc}:
    \begin{itemize}
	\item cdi/CDI.hh
	\item Mat\_State.hh
	\item DiscreteMaxPrinciple.hh
	\item Mat\_State.hh
	\item mc/OS\_Mesh.hh
	\item mc/Rage\_Mesh.hh
	\item mc/RZWedge\_Mesh.hh
	\item mc/Sphyramid\_Mesh.hh
	\item mc/Topology.hh
	\item mc/Comm\_Patterns.hh
	\item mc/Parallel\_Data\_Operator.hh
    \end{itemize}
 \item member method \texttt{run\_checks}:
     \begin{enumerate}
	\item Collect the number of cells and frequency groups
	\item Call \texttt{get\_Trad\_neighbors}
	\item Create Fleck factors
	\item Create vector of smart pointers to DiscreteMaxPrinciple objects
	\item Compute Planckian opcacities
	\item Initialize DiscreteMaxPrinciple calculations
     \end{enumerate}
 \item member method \texttt{get\_Trad\_neighbors} \\
	This method is specialized to each mesh because of the need to
communicate with neighboring cells.
	\begin{enumerate}
	\item Collect number of local cells
	\item Determine dimensionality, number of neighboring cells
	\item Populate vector of temperatures for each cell
	\end{enumerate}
\end{itemize}


\subsubsection{\texttt{DiscreteMaxPrinciple}}
\begin{itemize}
 \item Dependencies within \texttt{clubimc}
    \begin{itemize}
	\item math.h
	\item Opacity\_Builder\_Helper.hh
	\item Max\_Temperature\_Mesh\_Ops.hh
	\item Mat\_State.hh
	\item Global.hh
	\item mc/Constants.hh
	\item special\_functions/ExpInt.hh
	\item cdi/CDI.hh
    \end{itemize}
 \item member method \texttt{get\_inequality} \\
Returns the value of the DMP inequality.  If this is less than 0, a violation
is likely.
\end{itemize}

In implementing these classes into a use code, it is intended that an instance
of the \texttt{DMP\_Helper} class be generated and checks run
preceeding each new time step calculation. This instance will create the
required set of \texttt{DiscreteMaxPrinciple} instances in such a way that when
\texttt{DMP\_Helper} goes out of memory, so will the
\texttt{DiscreteMaxPrinciple} instances.  Based on the presence of any likely
violations, actions can be added to \texttt{DMP\_Helper} to either end the
process, raise a warning flag, or find a $\Delta_t$ that will not cause a
violation.
