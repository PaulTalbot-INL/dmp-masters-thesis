\section{Radiative Transport }
The transport equation governing general radiative transport is
\small
\begin{align}
 \frac{1}{c}\frac{\partial}{\partial t}I(r,\Omega,\nu,t) +
   \vec\Omega\cdot\nabla I(r,\Omega,\nu,t) + 
   &\sigma(r,\nu)I(r,\Omega,\nu,t) = \nonumber\\
   &j(r,\Omega,\nu,t) + 
   \frac{1}{4\pi c}\sigma_s(r,\nu)
     \int_{4\pi}I(r,\Omega,\nu,t)\ d\Omega \label{be_rad}.
\end{align}
\normalsize
In order of appearance, $c$ is the speed of light; $I$ is the radiation
intensity as a function of space ($r$), angle ($\Omega$), frequency ($\nu$), and
time ($t$); $\sigma$ is the opacity of the background material; $j$ is an
emission source; and $\sigma_s$ is a scattering term, describing the removal of
photons from 7-dimensional space-time $dr\ d\nu\ d\Omega\ dt$ into another such
space. Eq. \eqref{be_rad} serves as the starting point for this derivation of
the
discrete maximum principle.
% Separating each term distinctly, then,
% \small
% \begin{center}
%  \begin{tabular}{c|c|c|c|c}
%   Change Rate & Streaming & Abs. Loss & Emitted In & Scatter In \\
%   $\frac{1}{c}\frac{\partial}{\partial t}I(r,\Omega,\nu,t)$ & 
%     $\vec\Omega\cdot\nabla I(r,\Omega,\nu,t)$ & 
%     $\sigma(r,\nu)I(r,\Omega,\nu,t)$ &
%     $j(r,\Omega,\nu,t)$ & 
%     $\frac{1}{4\pi c}\sigma_s(r,\nu) \int_{4\pi}I(r,\Omega,\nu,t)d\Omega$
%  \end{tabular}
% \end{center}
% \normalsize

\section{Fleck Factor}
The bulk of this derivation will closely follow Wollaber, Larsen, and
Densmore's work in defining a discrete maximum principle \cite{WolLarDen}.  As
is typical to radiative transfer problems, the photon scattering opacity
$\sigma_s$ will be assumed small enough to cause the rightmost term in
\eqref{be_rad} to be negligible.  In addition, the time frame for photon
absorption into the material and subsequent emission is often very short, so
short as to be within a time step.  With sufficient coarseness in time step,
this interaction of absorption and re-emission at a different frequency is very
similar to a scattering interaction.  Thus, a factor $f$, the Fleck factor, can
be used to describe the level at which a particular problem can be represented
with scattering interactions.  The term $j(r,\Omega,\nu,t)$ then becomes two
terms, one for effective scatters and one representing absorptions that last
longer than a single time step.  This modifies the right side
 of Eq. \eqref{be_rad} as follows:
\begin{equation}
\mbox{RHS} = \frac{1-f}{2}\int_{\infty}\int_{4\pi}
    \sigma(r,\nu)I(r,\Omega,\nu,t)\ d\Omega\ d\nu +
  f\sigma(r,\nu)B(r,\nu),
\end{equation}
where the Planck function $B$ has been introduced, which is taken from the
solution of
the material equation associated with the radiation field equation as shown in
Eq. \eqref{be_mat}:
\begin{equation}
\frac{\partial}{\partial t}c_vT=\int_\infty\int_{4\pi}\sigma(r,\nu)(1-2\pi
  B(r,\nu))\ d\Omega\ d\nu \label{be_mat}.
\end{equation}
Here, the left side is the time-rate of change in the material
energy density.  Changing  \eqref{be_mat} to be time-differenced in
temperature gives the result arrived at by Fleck and Cummings \cite{FleckCumm}.

In many treatments of the radiative transfer equations, the
system is assumed to begin in total thermodynamic equilibrium, meaning the
material temperarture is initially in equilibrium with the radiation
temperature.
 It is the intent of this writeup to derive the discrete maximum principle
without making this assumption.