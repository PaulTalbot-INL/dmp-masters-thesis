\begin{center}
\section{Introduction}
\end{center}
\subsection{Thermal Radiative Transfer}\label{sec:intro_trt}
In general, three modes have been identified for heat transfer.  Conduction occurs when two static materials are adjacent to each other, and convection involves one material moving with respect to another.  The third mode, radiative heat transfer, is the result of photons generated by one material transporting to another and depositing some of their energy.  





For energies typical to most terrestrial problems such as 0.026 eV (30 degrees Celsius),  conduction and convection dominate heat transfer.  At higher energies like $10^3$ eV (11 million degrees Celsius),  as suggested by blackbody photon emission dependence on temperature, radiation becomes more important and can be the dominant form of heat transfer.  These kinds of extreme temperatures occur in many situations of interest, including fusion reactor startup and operation and celestial events such as supernovas.

%One particular reason the thermal radiative transfer equations are so interesting is their difficulty.  In general, the equations are as follows \cite{Pomraming}:
%\begin{subequations}\label{eqn:rtt}
%\begin{equation}\label{eqn:rtt1}
%\frac{1}{c}\frac{\partial I}{\partial t}(\mathbf{r},\mathbf{\Omega},\nu,t) + \mathbf{\Omega}\cdot\nabla I (\mathbf{r},\mathbf{\Omega},\nu,t)    
%    =\sigma_a(\mathbf{r},\mathbf{\Omega},t)\big[B(\nu,T)-I(\mathbf{r},\mathbf{\Omega},\nu,t)\big],
%\end{equation}
%\begin{equation}\label{eqn:rtt2}
%c_v(\mathbf{r},T)\frac{\partial T}{\partial t}(\mathbf{r},t)=
%        \int_0^\infty\int_{0}^{4\pi}\sigma_a(\mathbf{r},\nu',T)\big[(I(\mathbf{r},\Omega',\nu',t)
%            -B(\nu',T)\big]d\Omega'd\nu',
%\end{equation}
%\end{subequations}


\subsection{Previous Research}
In 1971, workers at the then-named Lawrence Radiation Laboratory proposed a new method for Monte Carlo treatment of the radiative heat transport equations, particularly targeting optically thick systems \cite{FleckCumm}.  As they pointed out, several attempts at applying Monte Carlo to the very nonlinear radiative heat transfer equations, but they were successful only largely for optically thin problems \cite{Fleck2}.  Fleck and Cummings' method served to semi-implicitly treat the radiative heat transfer equations, although, as they pointed out, some overheating occurred in certain choices of time and spatial steps, which is the focus of this work.  This method became commonly known as \gls{imc}.

While the \gls{imc} method quickly gained traction in radiative transport solutions, the undesirable instability depending on time and spatial step size led Larsen and Mercier to construct a maximum principle based on assuming temporal discretization.  While they did obtain a bounding equation for time step size, it proved very conservative, to the point of being impractical. 

More recently, Jeff Densmore and Ed Larsen \cite{DenLar} performed an asymptotic study on several Monte Carlo approaches to the radiative heat transfer equations: the Carter-Forest \cite{CarterForest}, N'Kaoua \cite{Nkaoua}, and Fleck and Cummings.  Since Fleck and Cummings' \gls{imc} approach is fundamental to this work, their findings are of interest.  In particular, Densmore and Larsen sought to test the equilibrium diffusion limit of the \gls{imc} method, meaning in an asymptotic analysis of the equations results in ``an accurate, time-discretized version of the equilibrium diffusion equation," then accurate solutions should be generated given an appropriate choice of time discretization.  They found that the \gls{imc} method did not have this limit, further highlighting the difficulties between temporal and spatial discretization and accurate solutions for the \gls{imc} equations.

Other attempts have also been made recently to find improved methods for radiative heat transfer.  At Los Alamos National Laboratory, Densmore, Urbatsch, Evans, and Buksas proposed a hybrid method using transport and diffusion called the Discrete Diffusion Monte Carlo (DDMC) \cite{ddmc}.  Particularly for diffusive problems, the number of interactions lead to a large number of small time steps in order to accurately capture the physics of the problem.  As with many hybrid methods, DDMC makes use of simplified diffusion solutions in regions within the mesh, replacing significant Monte Carlo particle tracking.  Also, the hybrid method switches between Monte Carlo transport in optically thin areas and the diffusion method for thick ones.  This provides another great tool in solving a variety of radiative heat transfer problems.

Another attempt to improve the \gls{imc} method involved adding an adaptive adjustment to the definition of the Fleck factor by McClarren and Urbatsch.  In general the Fleck factor includes a manually-selected $\alpha$ term, which is set to either 1/2 or 1.  McClaren shows how with slight redefinition of the Fleck factor, the \gls{imc} equations can result in less overheating.  We will make use of his developments in this work.

A masters thesis from Oregon State University \cite{long} concurrent with this work explored the possibility of using ``sub-time steps'' to update the temperature in a cell several times based on a small number of Monte Carlo particle runs in order to mitigate the overheating explored by Fleck and Cummings \cite{FleckCumm}, Larsen and Mercier \cite{LarsMerc}, and Wollaber, Larsen, and Densmore \cite{WolLarDen}.  This development proved effective at reducing overheating, and has an intriguing possibility of being combined with the present work to adaptively allow a code to correct for potential overheating problems.

Most recently, researchers at Los Alamos National Laboratory have developed a discrete maximum principle \cite{WolLarDen}, addressing the discrepancy between theory and practice brought up in \cite{LarsMerc}.  In addition to assuming temporal discretization, they included spatial discretization to develop an estimate for energy deposited in a cell as a function of space an time steps, which is then incorporated into a discrete maximum principle governing the maximum time step that will result in physically viable results for the problem.  This \gls{dmp} was demonstrated on a 1-dimensional Marshak wave problem, using a script separate from the simulation code.  This \gls{dmp} resulted in a bounding $\Delta_x, \Delta_t$ curve that closely matched experimental results.



\subsection{Research Objectives}
The objective of this work is threefold.

First, we adapt the existing \gls{dmp} to use multifrequency approximations.  Because the previous implementation of the \gls{dmp} used existing semi-analytic tools built into third-party codes, these approximations will demonstrate that the \gls{dmp} can be implemented directly into a wider variety of use codes.

Second, we remove the assumption of material and radiation field equilibrium initial conditions.  This change only requires a small adjustment to existing terms, and allows the \gls{dmp} to apply to a wider variety of physical systems.

Third, we demonstrate that, assuming linear superposition of energy deposition, the \gls{dmp} can be modified to consider temperature gradients across all boundaries of a cell instead of just the largest gradient.  While the existing method is excellent for single-dimension Marshak wave problems, this superposition assumption will allow a wider variety of problems to be protected by the \gls{dmp}.

Lastly, we show that none of these adjustment impair the operation of the \gls{dmp}, and the \gls{dmp} algorithm continues to predict overheating for a wide variety of problems.

\subsection{Outline}
TODO
