\section{Equations \ref{be_rad} and \ref{be_mat} to Equation \ref{time_start}}
\label{flcm}
This transition involves a significant amount of algebra and is the means by
which Fleck and Cummings \cite{FleckCumm} identified the Fleck factor to
linearize the thermal radiative transport equations. We begin
with repeating the two basic equations here:
\begin{subequations}
\begin{equation}
 \frac{1}{c}\frac{\partial}{\partial t}I(r,\Omega,\nu,t) +
   \vec\Omega\cdot\nabla I(r,\Omega,\nu,t) + \sigma(r,\nu)I(r,\Omega,\nu,t) =
  \sigma(r,\nu)B(r,\Omega,\nu)\tag{\ref{be_rad}}.
\end{equation}
\begin{equation}
\frac{\partial}{\partial
t}c_vT=\int_\infty\int_{4\pi}\sigma(r,\nu)\Big[I(r,\Omega,\nu,t)-2\pi
  B(r,\nu)\Big]\ d\Omega\ d\nu \tag{\ref{be_mat}}.
\end{equation}
\end{subequations}
Moving to a one-dimensional system in the $x$-direction, and removing the
dependency notation, Eq.\ \eqref{be_rad} becomes
\begin{equation}
 \frac{1}{c}\frac{\partial}{\partial t}I +
   \mu\frac{\partial I}{\partial x} + \sigma I = \frac{1}{2}\sigma
B\label{apx_rad_1d}.
\end{equation}
Analyzing the right-hand side of Eq.\ \eqref{apx_rad_1d}, $B$ is separable
into the normalized Planck spectrum $b$ and the equilibrium radiation energy
density variable $u_r$:
\[b \equiv \frac{15a}{\pi^4T^4}\ \frac{\nu^3}{e^{\nu/T}-1}, \]
\[u_r = aT^4, \]
\begin{equation}
B=bcu_r. \label{apx_newB}
\end{equation}
We then rearrange Eq.\ \eqref{be_mat}, in one dimension and without
dependencies,
\begin{equation}
\frac{\partial u_r}{\partial t} = \beta\iint\sigma Id\nu\ d\mu
    -\int_\infty\sigma Bd\nu, \label{apx_mat_1d}
\end{equation}
using the definitions in Eqs.\ \eqref{defs} as well as
\[\beta^{-1} \equiv \frac{\partial u_m}{\partial u_r},\]
\[u_m=c_vT,\]
\[\frac{\partial c_vT}{\partial t} = \frac{\partial u_m}{\partial t} = 
  \frac{1}{\beta}\frac{\partial u_r}{\partial u_t}\]

Substituting Eq.\ \eqref{apx_newB} into Eq.\ \eqref{apx_mat_1d}: 
\begin{equation}
\frac{\partial u_r}{\partial t}=\beta\iint\sigma I\ d\nu\ d\mu -
\beta c\sigma_pu_r
\end{equation}
and integrating
over a time step $t^n$ to $t^{n+1}$,
\[u_r^{n+1}-u_r^n= \int_{t^n}^{t^{n+1}}dt\ \bar\beta\iint\sigma Id\nu\ d\mu -
  c\int_{t^n}^{t^{n+1}}dt\beta\sigma_pu_r,\]
\begin{equation}
u_r^{n+1}-u_r^n\approx\bar\beta\Delta_t\iint\bar\sigma\bar Id\nu d\mu -
  \bar\beta\Delta_tc\bar\sigma_p\left[\alpha u_r^{n+1}+(1-\alpha)u_r^n\right].
\end{equation}
Here $\alpha$ is a term created by Fleck and Cummings to handle the
``implicitness'' of the problem.  $\alpha$ ranges from 1/2 to 1, with the
most implicit form being $\alpha=1$.
  Taking $\bar\sigma$  and, as a result, $\bar\sigma_p$ and $\bar\beta$ at the
beginning of the time step $t_n$ (acknowledging the significant approximation
this introduces) and solving for $\bar u_r = \alpha u_r^{n+1} + (1-\alpha)u_r^n$
gives
\begin{equation}
\bar u_r \approx \frac{\alpha\beta^n c\Delta_t}{1+\alpha\beta^n
c\Delta_t\sigma_p^n}\iint\sigma^n Id\nu d\mu +  cu_r^n\frac{1}{1+\alpha\beta^n
c\Delta_t\sigma_p^n}.\label{bar u}
\end{equation}
At this point we introduce the Fleck factor
\begin{equation}
f\equiv\frac{1}{1+\alpha\beta c\Delta_t\sigma_p},
\end{equation}
and substitute it into Eq.\ \eqref{bar u}:
\begin{equation}
\bar u_r=\frac{1-f^n}{\sigma_p^n}\iint\sigma^n Id\nu d\mu+cu_r^nf^n.
\end{equation}
Using this definition to expand the rightmost term in Eq.\ \eqref{apx_rad_1d}
at time step $t_n$,
\begin{align}
\frac{1}{2}\sigma B &\approx b^nc\bar u_r \nonumber \\
      &=\frac{1}{2}b^n\sigma\left[\frac{1-f^n}{\sigma_p^n}
           \iint\sigma^n Id\nu d\mu + cu_r^nf^n\right]\nonumber \\
      &= \frac{1-f^n}{2}\ \frac{\sigma^n b^n}{\sigma_p^n}\iint\sigma^n Id\nu
           d\mu + 2\pi f^n\sigma B^n, \label{apx solved}
\end{align}
where we note
\[bcu_r^n = 4\pi B^n.\]
Replacing Eq.\ \eqref{apx solved} into Eq.\
\eqref{apx_rad_1d} and setting $n=0$, we obtain Eq.\
\eqref{time_start_1}:
\begin{equation}
\frac{1}{c}\frac{\partial I}{\partial t} + \mu\frac{\partial I}{\partial x}
+\sigma I =
  \frac{1-f_0}{2}\frac{\sigma_0b_0}{\sigma_p}\int_0^\infty\int_{-1}^1
  \sigma_0 I\ d\mu\ d\nu + \sigma_0 f_0 2\pi B_0.\tag{\ref{time_start_1}}
\end{equation}