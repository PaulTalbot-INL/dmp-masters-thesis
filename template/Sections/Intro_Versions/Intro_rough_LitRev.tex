%=================%
% Section:        %
%    Introduction %
%=================%

\ResetSingleSpace

\begin{center}
{\textbf{
THE k-EIGENVALUE FROM MULTI-ENERGY GROUP NEUTRON TRANSPORT
IN BINARY STOCHASTIC MEDIA IN PLANAR GEOMETRY
}}
\end{center}

\ResetDoubleSpace

\vskip0.25in

\begin{center}
\section{INTRODUCTION}
\label{sec:Intro}
\end{center}

\setcounter{page}{1}
\thispagestyle{empty}

\vskip-0.1in

\noindent
	\indent In recent years, great interest has been generated in the radiation transport
	community on the interaction of particles with a stochastic or random
	heterogeneous background medium.  The background medium is considered stochastic
	or random for this class of problems since the precise location of any one of the constituents
	of the medium is not explicitly known.  In most nuclear engineering applications, the 
	background media has a known location and is fixed.  Further, it is common practice to treat
	material mixtures as homogenous at the atomic level by averaging the elemental 
	properties by their respective volume fractions.  This is a reasonable approximation when
	the microstructure of the mixture has a size less than a mean free path of the particle.
	Very accurate modeling techniques have been developed for particle transport and
	diffusion calculations for which the components of the background material are fixed and
	can be considered homogenous mixtures. 

\noindent
	\indent  All materials encountered in nature have some degree of heterogeneity. 
	The assumption that a medium is a homogenous mixture of its components can have 
	various effects on modeling the interaction of the particle with the medium.  As a particle
	traverses a homogenous medium, it encounters all of the constituents of the medium
	with equal probability during its lifetime.  In reality, this might not be the case.  The
	particle may encounter pockets of distinctly different material in the microstructure of the
	medium, causing profoundly different bulk particle flow effects	than would be predicted
	by a homogeneous model.  In these cases, the heterogeneous microstructure has to be
	given specific treatment.
	
\noindent
	\indent A large body of work now exists for modeling particle transport in stochastic media,
	but this area of research is still very active.  According to ~\cite{Pom:91b} and
	~\cite{Lew:93}  the original interest in these problems was generated in an attempt 
	to accurately describe radiative transfer in a capsule undergoing the implosion stage in
	inertial confinement fusion.  This is a system where there is radiative transfer through a
	medium composed of two fluids of drastically different densities, accelerated toward one
	another by the propagation front of the implosion shock wave~\cite{Cha:61}.  The lighter
	fluid pushes on and accelerates the denser fluid, causing a thin layer of turbulent mixing of
	these two immiscible materials by a Rayleigh-Taylor type instability.  If this layer is modeled as
	a homogeneous mixture, the radiation transport solutions in this medium include large error.  
	
\noindent
	\indent Since then, focus has been turned to other problems spanning a diverse set of disciplines
	with interest in particle transport, where a stochastic treatment of the background medium is
	necessary. The above example extends to larger scale modeling of astrophysical phenomena,
	such as core collapse supernovae, where the thermonuclear ignition after the death of a 
	massive star is still not well understood through available numerical models ~\cite{Mezz:04}. 
	Explicitly modeling heterogeneities may become necessary in fully understanding radiation
	transport and modeling errors in some already mature applications.  This may include the
	thermohydraulic and neutron density coupling in boiling water reactors where the neutrons
	traverse a bubbly flow (water-void mixtures).  Common materials used in radiation shielding
	may be treated more accurately as stochastic media, such as cement, where particles may
	interact with large regions of constituents with different neutron interaction probabilities.
	
\noindent
	\indent Explicit modeling of the microstructure is also necessary in global climate modeling.
	Clouds are heterogeneous mixtures of water vapor and ice.  Their composition, coverage,
	and vertical thickness dictate how much sunlight is reflected back into space and how much is
	absorbed and subsequently reemitted to the Earth's surface.  An understanding of cloud structure
	and the bulk radiative transport effects through them are fundamental to accurate atmospheric
	and oceanic circulation models.  Since radiative transport through a cloudy atmosphere still is not
	well understood, global climate models produce unacceptably erroneous solutions in the oceanic
	transport of heat, and predict global warming with uncertainty ranges which have not improved
	much in over two decades~\cite{Ste:02}.

\noindent
	\indent (FIX THIS PARAGRAPH WITH A GOOD REF) Radiation Transport must be preformed through in
	oceanography models of radiative transfer through murky ocean layers, or ocean layers that
	consists of water mixed with dissolved organic and inorganic materials.  The type, location,
	and amount of these dissolved materials are unknown precisely ~\cite{Mcco:04}.  Pebble
	bed reactors have at least three levels of stochasticity.  The fuel pebble itself is a random
	mixture of smaller fuel spheres in a graphite matrix used for moderation.  These pebbles are
	tightly packed and randomly mixed in the core region surrounded by gaseous coolant.  An
	additional layer of stochasticity is introduced after the reactor has been in operation for some
	time and the fuel in each pebble is depleted by the fission process at variable rates as they
	migrate through the core.  
	
\noindent
	\indent Radiation transport in nuclear reactor systems in which the fuel and moderator
	are a stochastic mixture is the focus of this thesis.  Quantifying the impact of randomness in
	these systems is important from a reactor operations and safety analysis standpoint. 
	The problems characteristic of nuclear reactor systems are eigenproblems, where the
	eigenvalue is an indication of how the neutron population is changing with successive
	generations, or the criticality, and the associated eigenvector, or neutron flux.  There are
	two types of eigenproblems commonly of interest in nuclear reactor analysis: transient and
	steady-state eigenproblems.  
	
\noindent
	\indent When evaluating nuclear reactor systems, changes in the reactivity of the system
	causes transient behavior of the neutron population in the limit of prompt neutron lifetimes
	in a typical power reactor (${<} {\approx}{10}^{-4}$ seconds) which constitute the bulk of
	the neutron population (~\cite{Dud:76}).  This does not include the ${<}$ 1 \% of delayed
	neutrons from the decay of fission fragments, photon-neutron reactions, or neutron
	absorption into nuclides leading to neutron emission.  Criticality can also be
	evaluated for steady state conditions in the limit of ``long'' times in a typical power reactor, 
	($\gg {10}^{-4}$ seconds), preceding some change in reactivity.  In the first case the
	nuclear engineer may be considering a problem where a given reactor is operating steadily
	at some initial power, when at a later time the reactor undergoes some change in the
	reactivity, causing a spike in the reactor power.  This can
	occur during normal operation through the movement of control rods, neutron �poisons�
	added to the reactor chemistry, or burnable poisons placed in the fuel for control purposes.
	The buildup and decay of fission products such as Iodine and Xenon or Promethium and
	Samarium, is common during startup and shutdown which adds or removes reactivity.
	Accident scenarios such as loss of coolant or loss of pressure can cause reactivity
	fluctuations for which robust safety systems have been designed in all modern nuclear
	reactors.  This type of reactor criticality scenario in the context of a stochastic background
	medium, is not the subject of this thesis, although some of these problems, such as
	burnable poisons with somewhat random location in a modern fuel pellet, are excellent
	candidates for potential stochastic media transport research.  The stochastic reactor
	criticality problem which is the subject of this thesis are those problems where there are
	given distributions of fuel and moderator loading for a reactor, and the system behavior is
	evaluated in the limit of ``long'' reactor times given random placement of the reactor material
	according to this distribution.

\noindent
	\indent This behavior of a stochastic nuclear reactor system in planar geometry is 
	investigated by the evaluation of the steady-state eigenproblem describing this 
	stochastic media system.  The neutron energies are restricted to exist only in certain
	energy ranges, for scattering events to cause only either no change in energy or a
	decrease in energy, as well as for scattering to be isotropic.  The remainder of the introduction
	is a literature review in the following section, which gives an overview of some of the
	published literature on radiation transport and diffusion for stochastic media to date, followed
	by an overview of the remaining sections of the thesis in Section \ref{sec:Intro-Overview}.

%====================================================================%
% SubSection:                        	                             %
%    Introduction: Literature Review %
%====================================================================%
\aboveSubSecSkip

\subsection{Literature Review}
\label{sec:Intro-LitRev}
	
\noindent
	\indent There has been two basic components of research in stochastic media: models and 
	benchmarks.   As discussed in Section \ref{sec:Intro}, the statical nature of a stochastic media
	transport problem enters through the random mixing of the components of the background
	medium.  For
	any given point in space in the mixture, a particle could encounter one of the components or an
	interface
	between components with probabilities governed by the mixing statistics.  Benchmark modeling of
	radiation transport in a stochastic media is straightforward and computationally expensive, while 
	models approximating benchmark solutions are complicated to derive but ideally should yield
	approximations which are accurate, simple, and computationally inexpensive.  The benchmark
	type calculations for stochastic
	media are done by populating the geometry with segments of each of the 
	constituents of some size
	according to known or assumed statistical distributions, and then performing a transport 
	calculation.  This is
	known as a single realization of the statistics.  Many realizations are performed, with each
	transport solution contributing to a statistical solution comprised of an average and
	higher order statistical moments for the ensemble.  At least thousands, if not millions, of
	realizations are performed to ensure that the distributions have been well sampled, and
	that statistical ensemble solutions have converged.  
	
\noindent
	\indent If the mixing statistics are known with any certainty, a particle transport solution can
	be generated for an individual realization of the medium as well as statistical ensemble
	quantities.  However, it is clear how the
	computational expense of this type of calculation can quickly become prohibitive as one
	contemplates calculations in two or three dimensions, considering~multi-material mixtures, with
	full temporal, energy, and angular dependence.  Since the derivation of an analytical solution is
	not
	possible except for very simple systems, approximate models must be derived.
	The accuracy of the models is then unknown except in certain limits.  Benchmark calculations
	provide an ``exact'' answer, acting as a metric against which models can be compared.
	Insight and understanding gained from benchmark calculations has been used to derive 
	models which predict ensemble quantities and improve the accuracy of their predictions.
	
\noindent
	\indent At the beginning of the research effort for radiation transport in stochastic media, it was the 
	deterministic models that were developed first from physical understanding of radiation transport in 
	certain types of simple stochastic media systems.  When approximations 
	were introduced into these early models to expand their capacity to predict the ensemble
	behavior of a wider variety of systems, benchmarks were then needed to ascertain the effect of
	these approximations on the accuracy of these models.  From that point both benchmarks and
	models started to be
	developed that included more physics and more complicated stochastic media systems.
	
\noindent 
	\indent This research has centered around two important types of particle transport
	problems:
	boundary value problems of the form $\underline{\underline{\mathbf{A}}} \
	\underline{\psi}=\underline{\mathbf{q}}$ and eigenproblems of the form
	$\underline{\underline{\mathbf{A}}}\ \underline{\psi}=\lambda\underline{\psi}$.  Even
	though the 
	subject of this thesis is eigenproblems for stochastic media transport, there is a rich history of
	research and acquired understanding regarding the nature of particle transport stochastic media
	from the analysis of boundary value
	problems which is necessary to discuss.  What has been learned for these problem types can be
	carried over for modeling and benchmarking for eigenproblems.

%==================%
% SubSubSection:   %
%    Boundary Value Problems in Stochastic Media %
%==================%
\subsubsection{Boundary Value Problems in Stochastic Media}
\label{sec:Intro-LitRev-BVP}  

\noindent
	\indent Most of the work to date regarding boundary value problems for particle transport in
	stochastic media has considered source free, isotropic scattering, monoenergetic
	transport in rod or planar geometry.  Rod geometry is meant to imply that particles are
	constrained to move along a line in either the positive or negative direction.  For isotropic
	scattering, particles can scatter into one of these directions with equal probability of
	(1/2)~\cite{Ada:89}.  Boundary value problems are driven by a particle flux incident on the 
	medium from its exterior.  Solutions have included the amount of transmission and reflection, as
	well as the particle flux profile in the interior of the medium.
	
\noindent
	\indent The most well known, easiest to implement, and consequently, most widely used model
	is known as atomic mix.  This approximation has been used continually in the past for fixed
	media applications and ignores the stochastic nature of material constituents.  The atomic mix
	approximation effectively homogenizes a mixture by volume averaging the characteristics of its
	constituents. This is a reasonable approximation for media with constituent sizes
	that are small compared to a mean free path of the transported particle~\cite{Mal:92}, or in the
	small correlation length limit.  For constituent sizes on the order of a particle mean free path or
	longer, the atomic mix approximation has been shown to overestimate the absorption as a
	general trend (\cite{Mal:92}, \cite{Mil:01}, and~\cite{Dav:04}).
	
\noindent 
	\indent Much of the
	work to date has been centered around deriving accurate models for benchmark calculations in 
	which the material segments or transition lengths are exponentially distributed or Markovian.  
	In~\cite{Pom:91b}, \cite{Pom:98}??, it is
	shown that a Markovian process is usually associated with a ``no-memory'' initial value problem 
	where, if the solution is known at some time $({t})$, then the solution is defined uniquely for
	some later
	time $({t^{'}})$, where $({t^{'} > t})$.  This is known as a no-memory process as it is unnecessary 
	to know a solution before $({t})$ to determine solutions at $({t})$ and $({t^{'}})$.
	
\noindent
	\indent The Boltzmann transport equation governing particle flow (see Section ??) in a purely
	absorbing medium where there is no time dependence is a 
	spatial Markovian process.  Such a system with assumed Markovian mixing statistics
	describes a joint Markov process, and the Liouville master
	equation applies.  For a binary mixture, this results in two coupled partial 
	differential equations for the stochastic transport problem with four unknowns: the volumetric
	angular flux and interface angular flux in each material.  Some relation is needed to close this
	system of 
	equations, which have come to be known as the~Levermore-Pomraning equations.  If the
	interface angular flux is set equal to the volumetric angular flux, the~Levermore-Pomraning 
	equations are exact
	for time independent problems in purely absorbing, binary stochastic media, with Markovian
	mixing statistics.  This is similar to the upwind closure scheme and will hereafter be referred to
	as the ``classic'' model.  Exact equations can be written for higher order statistical
	moments of the classic model such as the variance~\cite{Pom:96} for the same restricted class
	of problems.  Statistical moments greater than the average, like the variance, should be
	included as part of a complete and more meaningful statistical solution for particle transport in
	a stochastic medium.
	
\noindent 
	\indent The assumption of Markovian mixing statistics is not purely for academic purposes.  It
	has been shown that in the case of a binary stochastic medium in which disks or spheres are
	distributed in a background material, that the assumption of Markovian statistics in the
	background material is a valid one (\cite{Su:93} and \cite{Ols:03}).  Results 
	from~two-dimensional hydrodynamic calculations indicate that fluid segment lengths in a 
	turbulently mixed, binary stochastic media are Markovian in nature, as well as fragment sizes
	in rock~\cite{Pom:90b}.
	
\noindent 
	\indent Separate derivations have lead
	to this same coupled set of equations given by the classic model, including the method
	of smoothing, using the Liouville master equation, making the assumption that the particle
	trajectories are uncorrelated, using reactor noise techniques, and from a particle balance
	standpoint~\cite{Pom:91c}.  The classic model also goes to the atomic mix model when the
	transition lengths become vanishingly small, and goes to the standard transport equation
	for transition lengths which increase without bound~\cite{Mal:92}.    
	
\noindent
	\indent There are other models which are exact for binary stochastic media, but  in a limited
	scope or result in very complicated expressions.  Two independent sets of kinetic equations
	have been developed which independently and exactly describe time-dependent particle
	transport with arbitrary mixing statistics in the full three-dimensional setting including the scattering
	interaction~\cite{San:89}.  Unfortunately, each set of equations does not close resulting in an
	infinite system of coupled equations and, while providing a theoretical basis for construction of
	approximate models, are not amenable to application.  The theory of alternating renewal
	processes has yielded four coupled, exact transport equations also for a purely absorbing, binary
	stochastic medium without the restriction of Markovian statistics, but including Markovian statistics
	as a special case (\cite{Pom:89}, and \cite{Fri:90}, \cite{Pom:98}).  Exact equations for higher
	statistical moments can be written for this model as well.  An extension of this model has been
	recommended as an approximation to systems including time dependence and
	scattering~\cite{Pom:89}.  For a time-independent, purely scattering binary stochastic medium in
	rod geometry, an exact model has been formulated using, what has been called, the invariant
	imbedding method~\cite{Van:89} and using a more general formalism~\cite{Pom:88}.   
	
\noindent
	\indent These exact equations are for specialized cases and it is unknown how to extend them
	rigorously as a general model or they are not easily solvable as a general model.  Also, purely
	scattering media restricted to rod geometry or purely absorbing media or are almost never
	encountered in practical application, so more generalized models had to be derived and 
	investigated.  In many
	applications of particle transport, it is common for the background material to be modestly to highly
	scattering.  Therefore, to model radiation transport in realistic binary stochastic media, a model
	which is useful in application must also be accurate for scattering media.  As mentioned, the
	classic model is known to be exact in time-independent, purely absorbing media, but its ability
	to predict ensemble flux behavior for the boundary value problem was unknown if it was extended
	as an approximation to systems in which scattering was allowed.  When compared against early
	benchmark studies in which scattering was allowed in one or both of the materials in a binary 
	stochastic medium, the classic model was shown to underestimate the reflection probability and
	overestimate the transmission probability in both the rod and planar geometries in systems that 
	were sufficiently long (\cite{Ada:89}, \cite{Zuc:94}, and~\cite{Mil:01}).  It has been shown that the
	effective cross section for purely scattering media goes through a pronounced minimum in at
	some system length (\cite{Su:95}, \cite{Pom:??}, and~\cite{Van:89}).  This ``transmission
	window'' is not predicted by the classic model as the system length increases, but predicts that
	the effective cross section monotonically decreases as a function of system length.  Overall,
	the classic model gave solutions that were qualitatively correct.

\noindent
	\indent Different deterministic models have come about through research in employing a different
	closure for the~Levermore-Pomraning equations to include the scattering interaction.  It seems
	impossible to derive a set of closure equations relating the volumetric material angular flux to the
	interface material angular flux which result in an exact, general model~\cite{Su:95}.  More complex
	closure terms which include some scattering approximation has been the subject of much 
	research (\cite{Ada:89}, \cite{Pom:91c}, \cite{Su:94}, and~\cite{Su:95}).  

\noindent
	\indent Two additional low order closures were developed from a small 
	material transition length analysis (small segment lengths in one or both of the materials)
	in an attempt to develop a closure that goes to the atomic mix limit~\cite{Su:94}.  While these
	closures were more complex in approximating scattering, they were algebraic and thus considered
	low order closures.  Moreover, it was unclear how to determine these closure relationships in
	other than an \emph{ad hoc} way.  These two closures resulted in two models which are useful
	only for certain systems.  The new closures show only slight improvement to, or can be worse than,
	the classic model.  Both of the additional low order closures do not predict the transmission
	window and therefore none of the low order closure models (including the classic model) are well
	suited as a general model (\cite{Su:94} and~\cite{Su:95}).  
	
\noindent
	\indent A higher order closure was developed from a ``balance like'' equation for the angular
	flux at the interface of
	two materials in a binary stochastic medium~\cite{Pom:91b}.  This yielded two
	transport equations for the interface fluxes in each material coupled to the volumetric fluxes in
	each material described by the~Levermore-Pomraning equations; four coupled transport
	equations overall.  Pomraning called this model incorporating the higher order closure
	the ``Interface'' model.  Solving these four transport equations did 
	result in significant improvement over the classic model in approximating the reflection
	and transmission probabilities given in a particular benchmark when considering time
	independent, source-free, monoenergetic, isotropic scattering in rod geometry.  However, there
	seems to be no obvious way to extend this closure to general geometries and statistics.  The
	Interface model was tested against atomic mix, a decoupled form of the classic model
	(infinite correlation length model), the classic model, and three other models which
	were approximations to the classic model~\cite{Mal:92} using cross section data
	from~\cite{Ada:89} for rod geometry.  The overall conclusion of this study was that more
	complex models lead to more accurate solutions.
	
\noindent
	\indent One of the additional low order closures from~\cite{Su:94} was used in
	approximating~time-dependent grey radiation transport in a purely absorbing binary
	stochastic media where the transport was coupled to the matter.~\cite{Mil:01}.  The assumption
	of a~source-free, purely absorbing media was made with a linear form of the material
	temperature equation.  The binary stochastic medium was purely absorbing, but the coupling
	to the matter can be thought of as a scattering source, where the thermal radiation absorbed can
	be reemitted isotropically with some probability.  Atomic mix, the low order model
	from~\cite{Su:94}, and the classic model were compared against the benchmark.  As expected
	from the above
	discussion, the low order~\cite{Su:94} model produces a much more accurate answer than
	the classic model, since the closure takes into account some scattering.  The atomic mix
	approximation overestimated the absorption as predicted.  The cases considered
	were also in the small transition length limit which, from the above discussion, is the assumption
	under which the low order closure of~\cite{Su:94} was derived.  Therefore, this was 
	a good closure choice for the~Levermore-Pomraning equations, to obtain an accurate for this
	particular application.

\noindent
	\indent A linear combination of the Interface model and the classic model was also 
	developed~\cite{Su:95}.  It was found that the Interface model gave more accurate 
	reflection and transmission probabilities compared to the classic model, but sometimes
	could give unphysical flux profiles for some systems, which is a curious
	characteristic.  The Interface model has been 
	the only closure capable of predicting the transmission window with any accuracy.  The 
	classic model was linearly combined in an \emph{ad hoc} way such as to preserve the
	best qualities of each model.  For purely
	absorbing media the combined model goes to the classic model, which is exact.  For purely
	scattering systems, the combined model goes to the Interface model which is the best prediction
	of these systems.  This was tested for isotropic scattering, monoenergetic, systems in rod and
	planar geometry.  Numerical tests against benchmark solutions revealed that the combined
	model was both robust and yielded accurate results.  There were a few specific systems where
	the combined model can be inaccurate.  However, since the 
	combination of the classic and Interface model was done in an \emph{ad hoc} way, the
	model can be tuned to specific problems and obtain very accurate results.  Therefore, the
	combined model was recommended as a replacement for both the classic model and the 
	Interface model for source free, isotropically scattering, monoenergetic, binary mixtures in
	rod and planar geometry, but is not a general model~\cite{Su:95}. 
 	
\noindent
	\indent Some benchmark and modeling studies have been done for non-Markovian statistics.
	Considering six different mixing statistics in rod geometry, it was found that the different statistics
	do impact the ensemble flux profile~\cite{Su:93}.  For those statistics chosen in the evaluation,
	Markovian seemed to give a fair prediction of the ensemble average flux profile.  It was also noted
	that the standard deviation of the reflection and transmission probabilities were the greatest for
	Markovian statistics.  This result is logical since, of those statistical distribution considered,
	only the Markovian distribution allows an arbitrarily long segment, and has the greatest 
	standard deviation about the mean of any distribution.
	
\noindent
	\indent Similar findings were reported when considering eight different mixing statistics,
	including Markovian in planar geometry.  It was shown that the probability of transmission was
	more statistics dependent than the probability of reflection~\cite{Zuc:94} for systems of sufficient
	length.  This result seems reasonable for slabs which are long enough to contain many
	alternating segments of each material.  Particles would tend to encounter more of the possible
	variation present in the slab than those particles which partially penetrate the slab and reflect
	back out.  Another important result of this study is that different mixing statistics produce different
	results in the flux distribution.  Furthermore, the mean and variance of the statistical distribution
	describing the mixing statistics indicate the influence on the transport problem.  For the different
	unimodal statistical distributions evaluated, the material and ensemble average scalar fluxes
	are insensitive to statistical moments higher than the second, this being especially true in the
	case of well mixed systems (small transition lengths relative to the system length)
	(\cite{Lev:88} and \cite{Zuc:94}).  These studies also revealed that the mean segment length from
	a given distribution alone was insufficient statistical information for modeling purposes.  This has
	significant impact on modeling particle transport in stochastic media since one can choose any
	unimodal distribution with an equivalent mean and variance of the unimodal distribution being
	modeled, and get reasonably accurate results.  This is not necessarily the case for modeling
	particle transport in a stochastic medium with arbitrary mixing statistics.
	
\noindent
	\indent Models have been developed to predict the ensemble behavior of binary stochastic
	mixtures with non-Markovian statistics.  The aforementioned classic and Interface model were
	tested against a model based on averaging the integral form of the transport equation resulting
	in ``renewal-like'' equations for the ensemble average flux, aptly named the
	Renewal model.  A correction factor was applied to the Interface model as suggested
	in~\cite{Lev:88}.  All three models are exact for time-independent, purely absorbing stochastic
	mixtures with Markovian statistics.  The Renewal model is also exact for non-Markovian
	mixing statistics.  As expected from the above conversation, the classic model which was 
	derived strictly for Markovian mixtures is the least accurate of the models for non-Markovian
	mixtures.  The non-Markov corrected Interface model was found to be very accurate for the
	transmission and reflection probabilities, but could give unphysical flux profiles for some cases 
	as was the result of comparison to benchmarks for Markovian mixtures.  The Renewal model
	could robustly predict the qualitative behavior of the flux and the reflection and transmission
	probability~\cite{Zuc:94}.
	
\noindent
	\indent The correction factor suggested in~\cite{Lev:88} was also adapted to the combined
	model given in~\cite{Su:95} (discussed above) and tested as a prediction of a benchmark
	solutions for two kinds of mixing
	statistics other than Markovian.  The combined model was shown to be a robust model in the
	case of~non-Markovian mixing statistics.  The combined model lost some accuracy for small
	systems, but was shown to be more accurate for long systems than the classic
	model~\cite{Su:95}.

\noindent
	\indent Other models and benchmarks in stochastic media have been developed.  This includes
	a flux-limiting model for particle transport in a binary stochastic medium with Markovian mixing
	where a flux-limited diffusion approximation was made for the classic model resulting in two 
	coupled non-linear diffusion equations, which have been shown to be a more accurate diffusion
	approximation than classical isotropic diffusion in the non-stochastic case (\cite{Sam:91} and
	\cite{Pom:??}) (NEED MORE HERE)
	
\noindent
	\indent So far only deterministic models which have been developed for particle transport in
	stochastic media has been summarized.  There has been some benchmarking and modeling
	research done for particle transport in 
	stochastic media using the Monte Carlo method (see Section ??).  A chord length sampling (CLS) 
	and limited chord length sampling algorithm (LCLS) was adopted into a standard Monte Carlo
	calculation as a model for neutral particle transport in a two dimensional geometry for
	a binary stochastic medium of a fixed number of disks randomly placed in a matrix
	material (\cite{Don:03a} and \cite{Don:03b}.  This was performed as a boundary value problem
	with a monodirectional beam incident on one face of a rectangle defining the system boundaries.   
	The CLS algorithm samples from the mixing statistics of the respective materials to determine 
	whether or not a particle leaves a material or not, instead of determining if the particle enters a 
	different material by choosing the shortest distance between escape, collision, and distance to a 
	material boundary based on its current position and direction.  This models the stochastic mixture
	by assuming that the mixing statistics are a particle characteristic and not a material characteristic,
	sampling the mixing statistics in an approximated way as the particle traverses the medium.  This 
	can be much less computationally expensive than the benchmark calculation.
	This CLS algorithm was shown to be equivalent to the classic model.  The CLS algorithm assumes
	a Markovian transition length distribution in the matrix material and a distribution describing chord
	lengths in a disk for the disk material.  This assumption is not necessarily a reasonable one for
	particles which penetrate the surface of the disk, since this distribution will give an overestimation
	of the distance to escape from the disk to any particle whose position is in the disks interior.      
	
\noindent
	\indent The LCLS algorithm was developed which performed CLS in the 
	matrix material and traditional Monte Carlo in the disk material.  When considering reflection
	on the system incident flux boundary, and transmission on the other three system boundaries,
	both the CLS and LCLS algorithms were found to be within the statistics of the benchmark
	calculation for purely absorbing systems.  When scattering is introduced both algorithms become
	approximate with varying degrees of accuracy, with LCLS being the most accurate in general.  
	As in the case of the classic model, it was shown that the CLS algorithm underestimates
	reflection and overestimates transmission.  CLS and LCLS show a dramatic increase in
	computational efficiency relative to the Monte Carlo benchmark, except for a purely scattering
	case where CLS is much less efficient.
	
\noindent 
	\indent Another interesting Monte Carlo model was introduced which uses the Monte Carlo 
	method implementing Perlin noise functions as part of an alternative algorithm to the standard 
	Monte Carlo algorithm called the Woodcock algorithm~\cite{Tic:04}.  Like the CLS and LCLS
	algorithms the Woodcock algorithm assigns material characteristics to the particle reducing the 
	computational expense relative to the benchmark calculation.  However, in the case of the 
	Woodcock algorithm it is unnecessary to compute the distance to material interfaces.  When the
	algorithm determines a point of interaction, the Perlin noise function is evaluated to determine
	what material is assigned to that interaction point.  This model was shown to accurately
	predict experimentally measured fluorescent X-ray intensity from a granular Cu${_2}$S sample
	excited with a $^{238}$Pu source.  Furthermore, it was shown that this model can be extended to
	include models which could potentially simulate a broad variety of complicated stochastic media
	systems.  

\noindent
	\indent Modeling and benchmarking strategies have been summarized for boundary value
	problems for particle transport problems in stochastic media . The subject of this thesis is
	eigenvalue problems in binary stochastic media, but the research that has been performed
	for boundary value problems is valuable for future work in modeling and benchmarking
	eigenvalue problems.  A detailed summary and complete list of references of the research efforts
	to obtain exact models, closures for approximate models, as well as benchmarks
	for certain types of binary stochastic media systems are given in~\cite{Pom:98} and 
	in~\cite{Pom:??}. 
	   
\belowSubSecSkip

%==================%
% SubSubSection:   %
%    Eigenvalue Problems in Stochastic Media %
%==================%
\subsubsection{Eigenvalue Problems in Stochastic Media}
\label{sec:Intro-LitRev-Eig}
	
\noindent
	\indent Relative the bulk of research that has been done for the boundary value problem
	for stochastic media transport, only a small amount of research has been devoted to the 
	development of benchmarks and models concerning stochastic nuclear reactors.
	In the stochastic nuclear reactor, at least one component of the background mixture is a 
	multiplying component.  If a neutron interacts with such a component the reemission factor
	can be greater than unity from the fission process, where it would be unity for a pure scattering 
	component.  Much of this research has used diffusion theory as the basis for analysis, which
	does not adequately capture the complicated aspects of particle transport in a random
	multiplying medium.  Furthermore, the effect of randomness on the ensemble average criticality
	of a stochastic nuclear reactor can be ambiguous.  This is contrary to the particle flux, where 
	the ensemble average can easily be envisioned, as in the boundary value problem.
	
\noindent 
	\indent It is difficult, if not impossible, to infer the general effect of ``randomness'' on 
	``criticality''~\cite{Pom:91a}.  Using a perturbation about a non-stochastic reactor as a reference
	case while preserving the average total fuel loading and monenergetic diffusion theory, it was
	shown that general conclusions cannot be drawn regarding the effect of randomness on 
	criticality without specifically defining the indication of criticality and how randomness is introduced
	into the system.  Given the same perturbation from the reference
	reactor, it is possible to show that the ensemble average criticality of the system will increase,
	decrease, or stay constant, based on what is used as the indication of criticality, and how this
	criticality indicator is averaged.  The overall conclusion of this study was that neutron transport
	in a stochastic multiplying medium is far too complicated to make a general statement about
	the effect of random fluctuations in fuel density on the criticality of the system using a single
	ensemble averaged number.  Characterizations of a system as subcritical or supercritical may be
	too simple, and higher statistical moments of the criticality indicator may be necessary for a more
	meaningful solution. 

\noindent
	\indent  Steady-state, monoenergetic, planar geometry transport for a bare stochastic reactor has 
	also been considered~\cite{Jah:98}.  

\noindent
	\indent Consistent with~\cite{Jah:98}, when the eigenvalue is used as the specific indication of
	reactivity, it was found using the monoenergetic diffusion equation describing a Markovian
	binary stochastic reactor that the ensemble averaged reactivity is always greater than the 
	nonstochastic reference case~\cite{Pom:99}.  Randomness was introduced into the problem
	through fluctuations of the mean number of secondaries per collision about the ensemble average
	of this quantity.  This result was found to true only for rods of sufficient length and is dependent on
	what mode of the eigenvalue of interest.  If higher modes are considered the opposite conclusion
	can be drawn on the effect of randomness on criticality.  Thus, this conclusion was shown to be
	mode dependent.  Usually, the fundamental mode is the only mode considered, but the
	aforementioned conclusion is a complete one.
	


\noindent 
	\cite{Wil:00a}
	\indent Simulation based on the Feinberg-Galanin-Horning (FGH) method for heterogeneous
	reactors.  Used approximation for fuel as point sources and sinks.
	
	FGH uses a single group for the thermal neutrons and a slowing down kernel for the fast 
	neutrons.  Epithermal absorption is neglected.  Done for planar geometry, diffusion. 
	$^{235}$U fuel and graphite moderator where the fuel is considered thin plates separated by some 
	pitch.  And an approximate slowing down kernel, and segments of any width.
	FGH-
	where fuel elements are replaced by point, line, or plane sinks of thermal neutrons and 
	sources of fast neutrons resulting in a set of discrete linear equations from the diffusion equation
	giving the flux at each fuel element (a two group slowing down model) with random fuel placement
	from a uniform distribution (important!) or fuel elements with a fixed mean position in a regular
	lattice but can vary randomly around the mean, constant fuel and moderator loading.  With an
	ordered arrangement of fuel plates, found
	number of plates yielding critical size, then randomized the locations for these same number of
	plates.    
	
	age diffusion theory for the slowing down kernel.
	
	showed that if groupings of plates were near the reactor center, a maximum would occur, with
	some optimum plate grouping, determined by simulation.  Plates grouped at the reactor edges
	gave minimum criticality.  Showed a skewness in the probability distribution function of k-
	eigenvalues Showed that the greatest probability is that of a supercritical reactor
	(55.7 \%).   
	
	Also looked pdf of k-eigenvalue if plates were again arranged in the critical lattice and allowed
	to be shifted by a small perturbation about a mean value of 10 \% of the lattice pitch.  Resulted in 
	an almost Gaussian about criticality.  These experiments were not done analytically, but as 
	benchmarks.
	
\noindent
	\cite{Wil:00b}
	\indent The above FGH source-sink method was extended to include discrete resonance
	absorption assuming that the fuel lumps are additional sinks of neutrons assuming the resonance
	absorption is globally a weak effect.  Assumed that all resonance absorption takes place at one
	energy.  Infinite medium slowing down kernel.  
	
	1) Plate positions are fixed and values of enrichment are determined such that the reactor is 
	critical for reactors containing a variable number of fuel plates.
	2) Plates of identical enrichment are placed randomly with a uniform distribution in the reactor.
	3) Fixed positions with the enrichments of each plate chosen at random from a uniform distribution
	from 0 to 2y${_c}$, where y${_c}$ is the enrichment of each of the N plates from 0 to 16 yielding
	a critical reactor.  This yields an average enrichment that would result in a critical reactor on
	average.
	4) Both plate position and enrichment is random
	
	Does not know what type of error he is making as a function of realization number. 
	
	Conclusions: This paper extended the work of the above paper by including resonance 
	absorption and variable enrichment of the fuel plates.  The effect on the k${_{eff}}$ can 
	be studied for random plate positions with fixed enrichment, randomly enriched plates 
	with fixed positions, and both enrichment and plate position is random.  When the number of
	plates becomes large, the structure of the eigenvalue pdf becomes Gaussian-like, but never
	exactly Gaussian since there is a minimum and maximum value of the eigenvalue.  Research
	to date with one-speed diffusion theory for rod and planar geometry has shown that an increase
	in the ensemble average k-eigenvalue occurs for the stochastic reactor relative to a
	non-stochastic case.  When including resonance absorption into this model for a fuel mixture of
	natural uranium with some enrichment, the ensemble average k-eigenvalue is less than that
	of the non-stochastic case.  For fuel plates containing a mixture of ${^{235}}$U and ${^{27}}$Al
	there exists
	reactors which are subcritical and supercritical for a given number of plates in the reactor.  The 
	subcritical cases could possibly be attributed to realization error, since they are only slightly 
	subcritical and with a greater number of realizations could result in a critical or supercritical reactor.
	Therefore, any conclusions that can be drawn indeed depend upon how the randomness of the 
	system is defined and the fuel loading of the ensemble of reactor configurations.
	Variance in the ensemble average k is decreases as the number of plates in the reactor increases, 
	which is an expected result. 
	
\noindent
	\cite{Wil:02}
	\indent Like in \cite{Wil:00a}, considerations are for random placement of a fixed number of plates,
	or a small
	perturbation from a mean position.  This continues for this paper in which various numbers of 
	spherical fuel elements, approximated by point sources and sinks, are encased in a cube.  
	Diffusion theory is again used and spheres approximated by point sources but allowed to move
	around a box (this is the three-dimensional treatment).  The spheres are modeled as point sources
	and sinks, but with some radius in order to avoid a singularity in the definition of the FGH equation
	for the k${_{eff}}$ involving a Green's function.  The sphere coordinates are chosen from a uniform
	distribution.  All of the spheres are assumed to be the same size and consist of a homogeneous
	mixture of ${^{235}}$U and ${^{27}}$Al as in \cite{Wil:00b} but with random volume fraction of
	${^{235}}$U sampled from a uniform distribution.  A reflector was also used on all sides of the
	bounding box which was taken to be finite and infinite.  The following is studied:
	1) Fixed positions for spheres, random volume fractions
	2) Fixed volume fraction, random position for spheres
	3) Small random movements of spheres about some fixed position.  
	
	Disadvantages to the source-sink method:
	1) diffusion
	2) fuel lumps have to reasonably small so that they can be modeled by the Dirac delta function, 
	while the absorption of the lump is constant for the lattice whereas in reality it would depend on
	the number an location of the neighboring spheres to determine whether or not it could be
	regarded as isotropic over the sphere surface. 
	
	Just talk about how the above work was an extension of the 1-D stuff and that construction of 
	the PDF was done.  PDF construction and studies were not done relative to a non-stochastic
	critical reference case in this study.  Structure is examined for different numbers of spheres 
	populating the bounding box for different reactors with different reflectors (graphite or water). 
	
\noindent \cite{Wil:03a}
	\indent PDF from one speed, integral transport for small spheres in an infinite bulk medium where
	 the
	total cross sections in the spheres and matrix material is the same and small spheres in a 
	void or purely absorbing medium but with different total cross sections in the sphere and matrix
	material.  The spheres are contained in a bounding box.  This is just like what was done 
	above.  Diffusion theory results are also calculated to determine errors made in diffusion.  
	An infinitely reflected system is considered.  The spheres are a homogeneous mixture of 
	${^{235}}$U and some non-fissile material, where the enrichment of the sphere is variable, but
	keeping the total cross section of the sphere and background material the same, choosing v
	from a uniform random distribution between a maximum and minimum limit.  Total cross section
	is spatially constant for each realization, but variable enrichment.  Assumptions for the scattering
	term as volume averaged first flight collision probabilities were also made. 
	
	spheres with greater separation distance act more independently and thus the k-eigenvalue 
	solution has a greater variance.  
	
	As the number of spheres randomly placed increases, a continuous PDF starts to take shape
	become narrower as a large number of spheres makes sphere positions decreasingly random.
	
	For the problem with a constant cross section in the sphere and matrix material with random
	sphere positions, diffusion predicted the shape of the eigenvalue PDF well.  Diffusion
	did give the qualitatively correct shape of the PDF but was shifted in magnitude, always
	underpredicting the magnitude of the PDF.  Diffusion is not well suited for some problems like
	the spheres in the void background matrix.  PDF's from transport were constructed for the
	variable parameters showing the range of values the k-eigenvalue can be.  
	
\noindent  \cite{Wil:03b}
	\indent Source-Sink method is used with a graphite moderator and reflector of varying thickness
	with a variable number of plates.  As the reflector thickness increases, the ensemble 
	average k gradually increases and the relative standard deviation decreases.  The minimum
	eigenvale
	is show a greater increase in value than the maximum as the leakage is decreased, but the 
	PDF becomes narrower. 	Any fine structure in the PDF from preferred values of the k-eigenvalue
	are smoothed out with the introduction of a thicker reflector.  This was also shown for a 
	water reflector, but as in \cite{Wil:02}, the water reflector showed no fine structure as in the 
	graphite case because of the relative smallness of the slowing down and diffusion lengths 
	relative to the system size.  
	
\noindent \cite{Wil:01}
	\indent This paper compares a simulation of source-sink theory with a perturbation method used
	as a model for the source-sink theory benchmark.  Two group theory was again used for the 
	one-dimensional plate reactor.  For a given number of plates, the plate absorbing strength is 
	taken to be random sampled uniformly from an upper and lower limit moderated by graphite with
	fixed plate positions.  The mean absorbing strength makes the system of plates just critical.  Again
	as the number of plates increase, the shape becomes more Gaussian.  This particular study
	suggests that a random absorbing strength yields a mean k-eigenvalue less than unity, meaning it
	tends to decrease reactivity.  As the number of plates increases the variance decreases as 
	expected.  Perturbation theory applied to the source-sink method lends itself well as a prediction
	of the variance in this particular  problem.  As the randomness increases in the absorption
	strength, the perturbation approximation
	of the variance gets worse.  This result is expected.  The approximation gets better as the number
	of plates increases due to the fact that the distribution becomes more Gaussian-like as the
	number of plates increases. 
	
\noindent \cite{Wil:03c}
	\indent Obtained an analytical expression for the probability density function of the k-eigenvalue
	for an array of fissile spheres placed randomly from its initial position.  This was done for 
	spheres in a infinite background medium in which the total cross section in the sphere and
	moderator is the same and for spheres in a void.  Collision probabilities are used as interactions.
	The conclusion is that for a sufficient number of spheres, a Gaussian distribution well 
	represents the k-eigenvalue distribution.  Using integral transport theory from \cite{Wil:02} for 
	a variable number of spheres moved a small, uniformly random amount from their mean positions
	in a bounding box.  Perturbation theory is applied to this integral transport theory equation resulting
	in an equation for the perturbed k-eigenvalue.  The range of perturbation is examined to
	understand where the approximation can be used.  The assumption of constant cross section
	damps out much transport effects making diffusion a reasonable approximation (important!!) and 
	transport effects are indeed essential in many applications.  
	
\noindent \cite{Jah:98}
	\indent For planar geometry, time-independent, one-speed transport, three different components
	to a stochastic
	reactor were allowed to be randomly placed inside a constant width, bare slab.  The components 
	included a pure scatterer, pure absorber, and purely fissile materials.  The reference case 
	includes a uniform loading of these same components, which is critical.  Vacuum boundary
	conditions are the
	same in both the reference and stochastic case.  The composition is one of two types being a 
	homogeneous mixture of the aforementioned three components.  The probability of finding 
	either of the two materials at some random point in the slab is equal, namely, 1/2.  Both a binomial
	and
	exponential distribution of a variable number of even-integer segments are sampled from to 
	determine which material resides at any given point in the slab.  
	The slab is divided into a variable number of equal length segments of each component.
	Therefore, the exact number of perturbations is known based on the number of segments
	allowed in the binomial distribution.  When the component lengths are determined from the
	exponential distribution, the slab is terminated if either of the materials is exhausted, and 
	realizations run until the average k-eigenvalue converges.  The material loading in the 
	reactor is conserved by performing a calculation for each realization twice; one starting with 
	material 1 the other with material 0, since their respective probabilities are equal to 1/2.  This
	was done for different mean chord lengths as a fraction of the slab length.  
	
	The results of the two distribution types show that as the segment lengths decrease, the 
	ensemble average k-eigenvalue approach unity strictly from above for varying reemission values 
	varying from 1 to ${\nu}$.  
	
	This is to say that as the segment lengths become smaller that the eigenvalue approaches that
	of the homogenous reference reactor of unity.
		
	For these monoenergetic slabs of constant material loading, the 
	increase in the ensemble average k-eigenvalue relative to the critical reference reactor is a 
	function of the segment size and allowed reemission.  The departure from the reference 
	reactor is largest when the slab has little or no scattering interaction.  The reverse is true 
	when the slab is highly scattering.  The random structure impacts the ensemble averaged
	k-eigenvalue the greatest for highly absorbing slabs because particles travel a short distance
	on average.  When the slab is highly scattering, the effect of the random structure is reduced,
	since the scattering interaction will ``smooth out'' the random structure.  Thus, the reactor which
	is purely absorbing and the reactor which is highly scattering give the bounds for the maximum
	and minimum change in the eigenvalue.  It is unknown a priori if the ensemble average of the 
	k-eigenvalue from random realizations of these respective reactor types will be greater than, less
	than, or equal to the reference case.  Another way to look at 
	the overall result of an increase in the k-eigenvalue for random slabs is that atomic mix of 
	a random slab will result in an underestimation in the ensemble average k-eigenvalue, 
	which is important from a reactor safety standpoint. 
	
\noindent \cite{Jah:01}
	\indent Binomial statistics are again used in this study as in \cite{Jah:98}, giving fixed segment
	lengths of each of the two materials randomly distributed a reactor of fixed length, but 
	preserving the material loading.  The deterministic discrete ordinates code (ANISN) is used to
	evaluate the eigenvalue and gives excellent agreement with the Monte Carlo results of
	\cite{Jah:98}.  Detailed examination of the results shows that there is close to a fixed fraction
	of k-eigenvalues which contribute to the ensemble average which are less than critical while 
	predominantly the behavior is a supercriticality.  Those eigenvalues less than unity have a smaller
	negative negative reactivity addition than those above critical.  
	
	A general conclusion is that any distribution that does not distribute fissile isotopes to the boundary
	of the reactor will increase the ensemble averaged eigenvalue relative to the homogeneous
	reference reactor.  Also, it is apparent that segregation of the material species increases the
	average importance of interactions leading to particle production while decreasing the importance
	of interactions leading to particle loss.
	
	The work of \cite{Jah:98} has been extended to include multinomial distributions where a 
	cross section is discretized into m values.  A set of m equations can be generated for the bounding
	conditions of this type of system, allowing for realizations which preserve the material loading.  
	Same overall behavior as in the binomial case when using multinomial statistics.
	
\noindent
	\indent What hasn't been done:
	1) Transport in a true 1-D stochastic medium with variable energy dependence
	2) Modeling of this problem (we will look at atomic mix).
	
	
\belowSubSecSkip

%=====================================================================%
% SubSection:                        	                              %
%     Introduction: Thesis Overview %
%=====================================================================%
\subsection{Thesis Overview}
\label{sec:Intro-Overview}

\noindent
	\indent The remainder of the thesis has been written in sections to describe the steady-state,
	eigenproblem for stochastic media being solved.  The assumptions and restraints on
	the system studied are given with the corresponding equations, along with the input data and
	how it was generated.  The solutions obtained from the performed calculations are given and
	analyzed and the resulting conclusions drawn from those results are discussed Finally, what
	can be said about transport calculations for stochastic media in general from the examined
	system is discussed and work remains in this important and interesting research area.
	An outline of the rest of the thesis is given below, including a brief description of the content 
	of each section in general.
	
	\begin{enumerate}
		\item[Chapter II{.}]  This chapter introduces the general geometry, analytic Boltzmann 
		transport equation;
		a representation of the transport of neutral particles in any arbitrary medium.  The
		restrictions made on this equation are described to arrive at an analytic transport 
		equation for planar geometry.  The discretization scheme used to discretize each 
		of the independent variables in planar geometry transport is introduced, and the 
		iteration equations are provided.  The iteration techniques are then described 
		in which this discretized transport equation is solved, resulting in a neutron flux and 
		criticality solution.  A description of the derivation of the equations used in the diffusion
		synthetic acceleration used to improve the rate of convergence of the iterative solution
		procedure is then given.  
		\item[Chapter III{.}] This chapter describes different techniques in modeling stochastic
		media for radiation transport calculations.  The chosen Monte Carlo process for modeling
		the randomness of a stochastic media in this study is described.  A description is given of
		the system of interest, the different components of the system, and how the system is
		random.  Further, the distributions of chord lengths for this system is described and how
		these distributions were determined and sampled from is outlined.  Finally, the integration
		of this Monte Carlo modeling method for the described stochastic media system into the
		iterative deterministic transport calculation procedure of Chapter II is provided.  
	\end{enumerate}
	
	








